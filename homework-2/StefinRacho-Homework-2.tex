\documentclass[11pt]{article}

    \usepackage[breakable]{tcolorbox}
    \usepackage{parskip} % Stop auto-indenting (to mimic markdown behaviour)
    

    % Basic figure setup, for now with no caption control since it's done
    % automatically by Pandoc (which extracts ![](path) syntax from Markdown).
    \usepackage{graphicx}
    % Keep aspect ratio if custom image width or height is specified
    \setkeys{Gin}{keepaspectratio}
    % Maintain compatibility with old templates. Remove in nbconvert 6.0
    \let\Oldincludegraphics\includegraphics
    % Ensure that by default, figures have no caption (until we provide a
    % proper Figure object with a Caption API and a way to capture that
    % in the conversion process - todo).
    \usepackage{caption}
    \DeclareCaptionFormat{nocaption}{}
    \captionsetup{format=nocaption,aboveskip=0pt,belowskip=0pt}

    \usepackage{float}
    \floatplacement{figure}{H} % forces figures to be placed at the correct location
    \usepackage{xcolor} % Allow colors to be defined
    \usepackage{enumerate} % Needed for markdown enumerations to work
    \usepackage{geometry} % Used to adjust the document margins
    \usepackage{amsmath} % Equations
    \usepackage{amssymb} % Equations
    \usepackage{textcomp} % defines textquotesingle
    % Hack from http://tex.stackexchange.com/a/47451/13684:
    \AtBeginDocument{%
        \def\PYZsq{\textquotesingle}% Upright quotes in Pygmentized code
    }
    \usepackage{upquote} % Upright quotes for verbatim code
    \usepackage{eurosym} % defines \euro

    \usepackage{iftex}
    \ifPDFTeX
        \usepackage[T1]{fontenc}
        \IfFileExists{alphabeta.sty}{
              \usepackage{alphabeta}
          }{
              \usepackage[mathletters]{ucs}
              \usepackage[utf8x]{inputenc}
          }
    \else
        \usepackage{fontspec}
        \usepackage{unicode-math}
    \fi

    \usepackage{fancyvrb} % verbatim replacement that allows latex
    \usepackage{grffile} % extends the file name processing of package graphics
                         % to support a larger range
    \makeatletter % fix for old versions of grffile with XeLaTeX
    \@ifpackagelater{grffile}{2019/11/01}
    {
      % Do nothing on new versions
    }
    {
      \def\Gread@@xetex#1{%
        \IfFileExists{"\Gin@base".bb}%
        {\Gread@eps{\Gin@base.bb}}%
        {\Gread@@xetex@aux#1}%
      }
    }
    \makeatother
    \usepackage[Export]{adjustbox} % Used to constrain images to a maximum size
    \adjustboxset{max size={0.9\linewidth}{0.9\paperheight}}

    % The hyperref package gives us a pdf with properly built
    % internal navigation ('pdf bookmarks' for the table of contents,
    % internal cross-reference links, web links for URLs, etc.)
    \usepackage{hyperref}
    % The default LaTeX title has an obnoxious amount of whitespace. By default,
    % titling removes some of it. It also provides customization options.
    \usepackage{titling}
    \usepackage{longtable} % longtable support required by pandoc >1.10
    \usepackage{booktabs}  % table support for pandoc > 1.12.2
    \usepackage{array}     % table support for pandoc >= 2.11.3
    \usepackage{calc}      % table minipage width calculation for pandoc >= 2.11.1
    \usepackage[inline]{enumitem} % IRkernel/repr support (it uses the enumerate* environment)
    \usepackage[normalem]{ulem} % ulem is needed to support strikethroughs (\sout)
                                % normalem makes italics be italics, not underlines
    \usepackage{soul}      % strikethrough (\st) support for pandoc >= 3.0.0
    \usepackage{mathrsfs}
    

    
    % Colors for the hyperref package
    \definecolor{urlcolor}{rgb}{0,.145,.698}
    \definecolor{linkcolor}{rgb}{.71,0.21,0.01}
    \definecolor{citecolor}{rgb}{.12,.54,.11}

    % ANSI colors
    \definecolor{ansi-black}{HTML}{3E424D}
    \definecolor{ansi-black-intense}{HTML}{282C36}
    \definecolor{ansi-red}{HTML}{E75C58}
    \definecolor{ansi-red-intense}{HTML}{B22B31}
    \definecolor{ansi-green}{HTML}{00A250}
    \definecolor{ansi-green-intense}{HTML}{007427}
    \definecolor{ansi-yellow}{HTML}{DDB62B}
    \definecolor{ansi-yellow-intense}{HTML}{B27D12}
    \definecolor{ansi-blue}{HTML}{208FFB}
    \definecolor{ansi-blue-intense}{HTML}{0065CA}
    \definecolor{ansi-magenta}{HTML}{D160C4}
    \definecolor{ansi-magenta-intense}{HTML}{A03196}
    \definecolor{ansi-cyan}{HTML}{60C6C8}
    \definecolor{ansi-cyan-intense}{HTML}{258F8F}
    \definecolor{ansi-white}{HTML}{C5C1B4}
    \definecolor{ansi-white-intense}{HTML}{A1A6B2}
    \definecolor{ansi-default-inverse-fg}{HTML}{FFFFFF}
    \definecolor{ansi-default-inverse-bg}{HTML}{000000}

    % common color for the border for error outputs.
    \definecolor{outerrorbackground}{HTML}{FFDFDF}

    % commands and environments needed by pandoc snippets
    % extracted from the output of `pandoc -s`
    \providecommand{\tightlist}{%
      \setlength{\itemsep}{0pt}\setlength{\parskip}{0pt}}
    \DefineVerbatimEnvironment{Highlighting}{Verbatim}{commandchars=\\\{\}}
    % Add ',fontsize=\small' for more characters per line
    \newenvironment{Shaded}{}{}
    \newcommand{\KeywordTok}[1]{\textcolor[rgb]{0.00,0.44,0.13}{\textbf{{#1}}}}
    \newcommand{\DataTypeTok}[1]{\textcolor[rgb]{0.56,0.13,0.00}{{#1}}}
    \newcommand{\DecValTok}[1]{\textcolor[rgb]{0.25,0.63,0.44}{{#1}}}
    \newcommand{\BaseNTok}[1]{\textcolor[rgb]{0.25,0.63,0.44}{{#1}}}
    \newcommand{\FloatTok}[1]{\textcolor[rgb]{0.25,0.63,0.44}{{#1}}}
    \newcommand{\CharTok}[1]{\textcolor[rgb]{0.25,0.44,0.63}{{#1}}}
    \newcommand{\StringTok}[1]{\textcolor[rgb]{0.25,0.44,0.63}{{#1}}}
    \newcommand{\CommentTok}[1]{\textcolor[rgb]{0.38,0.63,0.69}{\textit{{#1}}}}
    \newcommand{\OtherTok}[1]{\textcolor[rgb]{0.00,0.44,0.13}{{#1}}}
    \newcommand{\AlertTok}[1]{\textcolor[rgb]{1.00,0.00,0.00}{\textbf{{#1}}}}
    \newcommand{\FunctionTok}[1]{\textcolor[rgb]{0.02,0.16,0.49}{{#1}}}
    \newcommand{\RegionMarkerTok}[1]{{#1}}
    \newcommand{\ErrorTok}[1]{\textcolor[rgb]{1.00,0.00,0.00}{\textbf{{#1}}}}
    \newcommand{\NormalTok}[1]{{#1}}

    % Additional commands for more recent versions of Pandoc
    \newcommand{\ConstantTok}[1]{\textcolor[rgb]{0.53,0.00,0.00}{{#1}}}
    \newcommand{\SpecialCharTok}[1]{\textcolor[rgb]{0.25,0.44,0.63}{{#1}}}
    \newcommand{\VerbatimStringTok}[1]{\textcolor[rgb]{0.25,0.44,0.63}{{#1}}}
    \newcommand{\SpecialStringTok}[1]{\textcolor[rgb]{0.73,0.40,0.53}{{#1}}}
    \newcommand{\ImportTok}[1]{{#1}}
    \newcommand{\DocumentationTok}[1]{\textcolor[rgb]{0.73,0.13,0.13}{\textit{{#1}}}}
    \newcommand{\AnnotationTok}[1]{\textcolor[rgb]{0.38,0.63,0.69}{\textbf{\textit{{#1}}}}}
    \newcommand{\CommentVarTok}[1]{\textcolor[rgb]{0.38,0.63,0.69}{\textbf{\textit{{#1}}}}}
    \newcommand{\VariableTok}[1]{\textcolor[rgb]{0.10,0.09,0.49}{{#1}}}
    \newcommand{\ControlFlowTok}[1]{\textcolor[rgb]{0.00,0.44,0.13}{\textbf{{#1}}}}
    \newcommand{\OperatorTok}[1]{\textcolor[rgb]{0.40,0.40,0.40}{{#1}}}
    \newcommand{\BuiltInTok}[1]{{#1}}
    \newcommand{\ExtensionTok}[1]{{#1}}
    \newcommand{\PreprocessorTok}[1]{\textcolor[rgb]{0.74,0.48,0.00}{{#1}}}
    \newcommand{\AttributeTok}[1]{\textcolor[rgb]{0.49,0.56,0.16}{{#1}}}
    \newcommand{\InformationTok}[1]{\textcolor[rgb]{0.38,0.63,0.69}{\textbf{\textit{{#1}}}}}
    \newcommand{\WarningTok}[1]{\textcolor[rgb]{0.38,0.63,0.69}{\textbf{\textit{{#1}}}}}


    % Define a nice break command that doesn't care if a line doesn't already
    % exist.
    \def\br{\hspace*{\fill} \\* }
    % Math Jax compatibility definitions
    \def\gt{>}
    \def\lt{<}
    \let\Oldtex\TeX
    \let\Oldlatex\LaTeX
    \renewcommand{\TeX}{\textrm{\Oldtex}}
    \renewcommand{\LaTeX}{\textrm{\Oldlatex}}
    % Document parameters
    % Document title
    \title{StefinRacho-Homework-2}
    
    
    
    
    
    
    
% Pygments definitions
\makeatletter
\def\PY@reset{\let\PY@it=\relax \let\PY@bf=\relax%
    \let\PY@ul=\relax \let\PY@tc=\relax%
    \let\PY@bc=\relax \let\PY@ff=\relax}
\def\PY@tok#1{\csname PY@tok@#1\endcsname}
\def\PY@toks#1+{\ifx\relax#1\empty\else%
    \PY@tok{#1}\expandafter\PY@toks\fi}
\def\PY@do#1{\PY@bc{\PY@tc{\PY@ul{%
    \PY@it{\PY@bf{\PY@ff{#1}}}}}}}
\def\PY#1#2{\PY@reset\PY@toks#1+\relax+\PY@do{#2}}

\@namedef{PY@tok@w}{\def\PY@tc##1{\textcolor[rgb]{0.73,0.73,0.73}{##1}}}
\@namedef{PY@tok@c}{\let\PY@it=\textit\def\PY@tc##1{\textcolor[rgb]{0.24,0.48,0.48}{##1}}}
\@namedef{PY@tok@cp}{\def\PY@tc##1{\textcolor[rgb]{0.61,0.40,0.00}{##1}}}
\@namedef{PY@tok@k}{\let\PY@bf=\textbf\def\PY@tc##1{\textcolor[rgb]{0.00,0.50,0.00}{##1}}}
\@namedef{PY@tok@kp}{\def\PY@tc##1{\textcolor[rgb]{0.00,0.50,0.00}{##1}}}
\@namedef{PY@tok@kt}{\def\PY@tc##1{\textcolor[rgb]{0.69,0.00,0.25}{##1}}}
\@namedef{PY@tok@o}{\def\PY@tc##1{\textcolor[rgb]{0.40,0.40,0.40}{##1}}}
\@namedef{PY@tok@ow}{\let\PY@bf=\textbf\def\PY@tc##1{\textcolor[rgb]{0.67,0.13,1.00}{##1}}}
\@namedef{PY@tok@nb}{\def\PY@tc##1{\textcolor[rgb]{0.00,0.50,0.00}{##1}}}
\@namedef{PY@tok@nf}{\def\PY@tc##1{\textcolor[rgb]{0.00,0.00,1.00}{##1}}}
\@namedef{PY@tok@nc}{\let\PY@bf=\textbf\def\PY@tc##1{\textcolor[rgb]{0.00,0.00,1.00}{##1}}}
\@namedef{PY@tok@nn}{\let\PY@bf=\textbf\def\PY@tc##1{\textcolor[rgb]{0.00,0.00,1.00}{##1}}}
\@namedef{PY@tok@ne}{\let\PY@bf=\textbf\def\PY@tc##1{\textcolor[rgb]{0.80,0.25,0.22}{##1}}}
\@namedef{PY@tok@nv}{\def\PY@tc##1{\textcolor[rgb]{0.10,0.09,0.49}{##1}}}
\@namedef{PY@tok@no}{\def\PY@tc##1{\textcolor[rgb]{0.53,0.00,0.00}{##1}}}
\@namedef{PY@tok@nl}{\def\PY@tc##1{\textcolor[rgb]{0.46,0.46,0.00}{##1}}}
\@namedef{PY@tok@ni}{\let\PY@bf=\textbf\def\PY@tc##1{\textcolor[rgb]{0.44,0.44,0.44}{##1}}}
\@namedef{PY@tok@na}{\def\PY@tc##1{\textcolor[rgb]{0.41,0.47,0.13}{##1}}}
\@namedef{PY@tok@nt}{\let\PY@bf=\textbf\def\PY@tc##1{\textcolor[rgb]{0.00,0.50,0.00}{##1}}}
\@namedef{PY@tok@nd}{\def\PY@tc##1{\textcolor[rgb]{0.67,0.13,1.00}{##1}}}
\@namedef{PY@tok@s}{\def\PY@tc##1{\textcolor[rgb]{0.73,0.13,0.13}{##1}}}
\@namedef{PY@tok@sd}{\let\PY@it=\textit\def\PY@tc##1{\textcolor[rgb]{0.73,0.13,0.13}{##1}}}
\@namedef{PY@tok@si}{\let\PY@bf=\textbf\def\PY@tc##1{\textcolor[rgb]{0.64,0.35,0.47}{##1}}}
\@namedef{PY@tok@se}{\let\PY@bf=\textbf\def\PY@tc##1{\textcolor[rgb]{0.67,0.36,0.12}{##1}}}
\@namedef{PY@tok@sr}{\def\PY@tc##1{\textcolor[rgb]{0.64,0.35,0.47}{##1}}}
\@namedef{PY@tok@ss}{\def\PY@tc##1{\textcolor[rgb]{0.10,0.09,0.49}{##1}}}
\@namedef{PY@tok@sx}{\def\PY@tc##1{\textcolor[rgb]{0.00,0.50,0.00}{##1}}}
\@namedef{PY@tok@m}{\def\PY@tc##1{\textcolor[rgb]{0.40,0.40,0.40}{##1}}}
\@namedef{PY@tok@gh}{\let\PY@bf=\textbf\def\PY@tc##1{\textcolor[rgb]{0.00,0.00,0.50}{##1}}}
\@namedef{PY@tok@gu}{\let\PY@bf=\textbf\def\PY@tc##1{\textcolor[rgb]{0.50,0.00,0.50}{##1}}}
\@namedef{PY@tok@gd}{\def\PY@tc##1{\textcolor[rgb]{0.63,0.00,0.00}{##1}}}
\@namedef{PY@tok@gi}{\def\PY@tc##1{\textcolor[rgb]{0.00,0.52,0.00}{##1}}}
\@namedef{PY@tok@gr}{\def\PY@tc##1{\textcolor[rgb]{0.89,0.00,0.00}{##1}}}
\@namedef{PY@tok@ge}{\let\PY@it=\textit}
\@namedef{PY@tok@gs}{\let\PY@bf=\textbf}
\@namedef{PY@tok@ges}{\let\PY@bf=\textbf\let\PY@it=\textit}
\@namedef{PY@tok@gp}{\let\PY@bf=\textbf\def\PY@tc##1{\textcolor[rgb]{0.00,0.00,0.50}{##1}}}
\@namedef{PY@tok@go}{\def\PY@tc##1{\textcolor[rgb]{0.44,0.44,0.44}{##1}}}
\@namedef{PY@tok@gt}{\def\PY@tc##1{\textcolor[rgb]{0.00,0.27,0.87}{##1}}}
\@namedef{PY@tok@err}{\def\PY@bc##1{{\setlength{\fboxsep}{\string -\fboxrule}\fcolorbox[rgb]{1.00,0.00,0.00}{1,1,1}{\strut ##1}}}}
\@namedef{PY@tok@kc}{\let\PY@bf=\textbf\def\PY@tc##1{\textcolor[rgb]{0.00,0.50,0.00}{##1}}}
\@namedef{PY@tok@kd}{\let\PY@bf=\textbf\def\PY@tc##1{\textcolor[rgb]{0.00,0.50,0.00}{##1}}}
\@namedef{PY@tok@kn}{\let\PY@bf=\textbf\def\PY@tc##1{\textcolor[rgb]{0.00,0.50,0.00}{##1}}}
\@namedef{PY@tok@kr}{\let\PY@bf=\textbf\def\PY@tc##1{\textcolor[rgb]{0.00,0.50,0.00}{##1}}}
\@namedef{PY@tok@bp}{\def\PY@tc##1{\textcolor[rgb]{0.00,0.50,0.00}{##1}}}
\@namedef{PY@tok@fm}{\def\PY@tc##1{\textcolor[rgb]{0.00,0.00,1.00}{##1}}}
\@namedef{PY@tok@vc}{\def\PY@tc##1{\textcolor[rgb]{0.10,0.09,0.49}{##1}}}
\@namedef{PY@tok@vg}{\def\PY@tc##1{\textcolor[rgb]{0.10,0.09,0.49}{##1}}}
\@namedef{PY@tok@vi}{\def\PY@tc##1{\textcolor[rgb]{0.10,0.09,0.49}{##1}}}
\@namedef{PY@tok@vm}{\def\PY@tc##1{\textcolor[rgb]{0.10,0.09,0.49}{##1}}}
\@namedef{PY@tok@sa}{\def\PY@tc##1{\textcolor[rgb]{0.73,0.13,0.13}{##1}}}
\@namedef{PY@tok@sb}{\def\PY@tc##1{\textcolor[rgb]{0.73,0.13,0.13}{##1}}}
\@namedef{PY@tok@sc}{\def\PY@tc##1{\textcolor[rgb]{0.73,0.13,0.13}{##1}}}
\@namedef{PY@tok@dl}{\def\PY@tc##1{\textcolor[rgb]{0.73,0.13,0.13}{##1}}}
\@namedef{PY@tok@s2}{\def\PY@tc##1{\textcolor[rgb]{0.73,0.13,0.13}{##1}}}
\@namedef{PY@tok@sh}{\def\PY@tc##1{\textcolor[rgb]{0.73,0.13,0.13}{##1}}}
\@namedef{PY@tok@s1}{\def\PY@tc##1{\textcolor[rgb]{0.73,0.13,0.13}{##1}}}
\@namedef{PY@tok@mb}{\def\PY@tc##1{\textcolor[rgb]{0.40,0.40,0.40}{##1}}}
\@namedef{PY@tok@mf}{\def\PY@tc##1{\textcolor[rgb]{0.40,0.40,0.40}{##1}}}
\@namedef{PY@tok@mh}{\def\PY@tc##1{\textcolor[rgb]{0.40,0.40,0.40}{##1}}}
\@namedef{PY@tok@mi}{\def\PY@tc##1{\textcolor[rgb]{0.40,0.40,0.40}{##1}}}
\@namedef{PY@tok@il}{\def\PY@tc##1{\textcolor[rgb]{0.40,0.40,0.40}{##1}}}
\@namedef{PY@tok@mo}{\def\PY@tc##1{\textcolor[rgb]{0.40,0.40,0.40}{##1}}}
\@namedef{PY@tok@ch}{\let\PY@it=\textit\def\PY@tc##1{\textcolor[rgb]{0.24,0.48,0.48}{##1}}}
\@namedef{PY@tok@cm}{\let\PY@it=\textit\def\PY@tc##1{\textcolor[rgb]{0.24,0.48,0.48}{##1}}}
\@namedef{PY@tok@cpf}{\let\PY@it=\textit\def\PY@tc##1{\textcolor[rgb]{0.24,0.48,0.48}{##1}}}
\@namedef{PY@tok@c1}{\let\PY@it=\textit\def\PY@tc##1{\textcolor[rgb]{0.24,0.48,0.48}{##1}}}
\@namedef{PY@tok@cs}{\let\PY@it=\textit\def\PY@tc##1{\textcolor[rgb]{0.24,0.48,0.48}{##1}}}

\def\PYZbs{\char`\\}
\def\PYZus{\char`\_}
\def\PYZob{\char`\{}
\def\PYZcb{\char`\}}
\def\PYZca{\char`\^}
\def\PYZam{\char`\&}
\def\PYZlt{\char`\<}
\def\PYZgt{\char`\>}
\def\PYZsh{\char`\#}
\def\PYZpc{\char`\%}
\def\PYZdl{\char`\$}
\def\PYZhy{\char`\-}
\def\PYZsq{\char`\'}
\def\PYZdq{\char`\"}
\def\PYZti{\char`\~}
% for compatibility with earlier versions
\def\PYZat{@}
\def\PYZlb{[}
\def\PYZrb{]}
\makeatother


    % For linebreaks inside Verbatim environment from package fancyvrb.
    \makeatletter
        \newbox\Wrappedcontinuationbox
        \newbox\Wrappedvisiblespacebox
        \newcommand*\Wrappedvisiblespace {\textcolor{red}{\textvisiblespace}}
        \newcommand*\Wrappedcontinuationsymbol {\textcolor{red}{\llap{\tiny$\m@th\hookrightarrow$}}}
        \newcommand*\Wrappedcontinuationindent {3ex }
        \newcommand*\Wrappedafterbreak {\kern\Wrappedcontinuationindent\copy\Wrappedcontinuationbox}
        % Take advantage of the already applied Pygments mark-up to insert
        % potential linebreaks for TeX processing.
        %        {, <, #, %, $, ' and ": go to next line.
        %        _, }, ^, &, >, - and ~: stay at end of broken line.
        % Use of \textquotesingle for straight quote.
        \newcommand*\Wrappedbreaksatspecials {%
            \def\PYGZus{\discretionary{\char`\_}{\Wrappedafterbreak}{\char`\_}}%
            \def\PYGZob{\discretionary{}{\Wrappedafterbreak\char`\{}{\char`\{}}%
            \def\PYGZcb{\discretionary{\char`\}}{\Wrappedafterbreak}{\char`\}}}%
            \def\PYGZca{\discretionary{\char`\^}{\Wrappedafterbreak}{\char`\^}}%
            \def\PYGZam{\discretionary{\char`\&}{\Wrappedafterbreak}{\char`\&}}%
            \def\PYGZlt{\discretionary{}{\Wrappedafterbreak\char`\<}{\char`\<}}%
            \def\PYGZgt{\discretionary{\char`\>}{\Wrappedafterbreak}{\char`\>}}%
            \def\PYGZsh{\discretionary{}{\Wrappedafterbreak\char`\#}{\char`\#}}%
            \def\PYGZpc{\discretionary{}{\Wrappedafterbreak\char`\%}{\char`\%}}%
            \def\PYGZdl{\discretionary{}{\Wrappedafterbreak\char`\$}{\char`\$}}%
            \def\PYGZhy{\discretionary{\char`\-}{\Wrappedafterbreak}{\char`\-}}%
            \def\PYGZsq{\discretionary{}{\Wrappedafterbreak\textquotesingle}{\textquotesingle}}%
            \def\PYGZdq{\discretionary{}{\Wrappedafterbreak\char`\"}{\char`\"}}%
            \def\PYGZti{\discretionary{\char`\~}{\Wrappedafterbreak}{\char`\~}}%
        }
        % Some characters . , ; ? ! / are not pygmentized.
        % This macro makes them "active" and they will insert potential linebreaks
        \newcommand*\Wrappedbreaksatpunct {%
            \lccode`\~`\.\lowercase{\def~}{\discretionary{\hbox{\char`\.}}{\Wrappedafterbreak}{\hbox{\char`\.}}}%
            \lccode`\~`\,\lowercase{\def~}{\discretionary{\hbox{\char`\,}}{\Wrappedafterbreak}{\hbox{\char`\,}}}%
            \lccode`\~`\;\lowercase{\def~}{\discretionary{\hbox{\char`\;}}{\Wrappedafterbreak}{\hbox{\char`\;}}}%
            \lccode`\~`\:\lowercase{\def~}{\discretionary{\hbox{\char`\:}}{\Wrappedafterbreak}{\hbox{\char`\:}}}%
            \lccode`\~`\?\lowercase{\def~}{\discretionary{\hbox{\char`\?}}{\Wrappedafterbreak}{\hbox{\char`\?}}}%
            \lccode`\~`\!\lowercase{\def~}{\discretionary{\hbox{\char`\!}}{\Wrappedafterbreak}{\hbox{\char`\!}}}%
            \lccode`\~`\/\lowercase{\def~}{\discretionary{\hbox{\char`\/}}{\Wrappedafterbreak}{\hbox{\char`\/}}}%
            \catcode`\.\active
            \catcode`\,\active
            \catcode`\;\active
            \catcode`\:\active
            \catcode`\?\active
            \catcode`\!\active
            \catcode`\/\active
            \lccode`\~`\~
        }
    \makeatother

    \let\OriginalVerbatim=\Verbatim
    \makeatletter
    \renewcommand{\Verbatim}[1][1]{%
        %\parskip\z@skip
        \sbox\Wrappedcontinuationbox {\Wrappedcontinuationsymbol}%
        \sbox\Wrappedvisiblespacebox {\FV@SetupFont\Wrappedvisiblespace}%
        \def\FancyVerbFormatLine ##1{\hsize\linewidth
            \vtop{\raggedright\hyphenpenalty\z@\exhyphenpenalty\z@
                \doublehyphendemerits\z@\finalhyphendemerits\z@
                \strut ##1\strut}%
        }%
        % If the linebreak is at a space, the latter will be displayed as visible
        % space at end of first line, and a continuation symbol starts next line.
        % Stretch/shrink are however usually zero for typewriter font.
        \def\FV@Space {%
            \nobreak\hskip\z@ plus\fontdimen3\font minus\fontdimen4\font
            \discretionary{\copy\Wrappedvisiblespacebox}{\Wrappedafterbreak}
            {\kern\fontdimen2\font}%
        }%

        % Allow breaks at special characters using \PYG... macros.
        \Wrappedbreaksatspecials
        % Breaks at punctuation characters . , ; ? ! and / need catcode=\active
        \OriginalVerbatim[#1,codes*=\Wrappedbreaksatpunct]%
    }
    \makeatother

    % Exact colors from NB
    \definecolor{incolor}{HTML}{303F9F}
    \definecolor{outcolor}{HTML}{D84315}
    \definecolor{cellborder}{HTML}{CFCFCF}
    \definecolor{cellbackground}{HTML}{F7F7F7}

    % prompt
    \makeatletter
    \newcommand{\boxspacing}{\kern\kvtcb@left@rule\kern\kvtcb@boxsep}
    \makeatother
    \newcommand{\prompt}[4]{
        {\ttfamily\llap{{\color{#2}[#3]:\hspace{3pt}#4}}\vspace{-\baselineskip}}
    }
    

    
    % Prevent overflowing lines due to hard-to-break entities
    \sloppy
    % Setup hyperref package
    \hypersetup{
      breaklinks=true,  % so long urls are correctly broken across lines
      colorlinks=true,
      urlcolor=urlcolor,
      linkcolor=linkcolor,
      citecolor=citecolor,
      }
    % Slightly bigger margins than the latex defaults
    
    \geometry{verbose,tmargin=1in,bmargin=1in,lmargin=1in,rmargin=1in}
    
    

\begin{document}
    
    \maketitle
    
    

    
    \subsection{Homework 2}\label{homework-2}

\subsubsection{Due Friday, 9/20/2024}\label{due-friday-9202024}

\subsubsection{For each homework set:}\label{for-each-homework-set}

\subsubsection{Create a New iPython (Jupyter) notebook. Name the
notebook FirstAndLastName\_Homework2 and save it before you start
working}\label{create-a-new-ipython-jupyter-notebook.-name-the-notebook-firstandlastname_homework2-and-save-it-before-you-start-working}

\subsubsection{To submit, export or print your notebook as a pdf, with
all outputs visible. Upload both the pdf and a copy of your notebook
(.ipynb) in
Canvas.}\label{to-submit-export-or-print-your-notebook-as-a-pdf-with-all-outputs-visible.-upload-both-the-pdf-and-a-copy-of-your-notebook-.ipynb-in-canvas.}

    \begin{enumerate}
\def\labelenumi{\arabic{enumi}.}
\item
  Exercise 2.8: Create two arrays a and b a = {[}1,2,3,4{]} b =
  {[}2,4,6,8{]} Compute the following:

  b/a + 1 b/(a+1) 1/a

  Verify these values by computing it by hand. (Either paper or using
  Markdown)
\end{enumerate}

    \begin{tcolorbox}[breakable, size=fbox, boxrule=1pt, pad at break*=1mm,colback=cellbackground, colframe=cellborder]
\prompt{In}{incolor}{3}{\boxspacing}
\begin{Verbatim}[commandchars=\\\{\}]
\PY{k+kn}{import} \PY{n+nn}{numpy} \PY{k}{as} \PY{n+nn}{np}
\PY{k+kn}{import} \PY{n+nn}{timeit}
\end{Verbatim}
\end{tcolorbox}

    \begin{tcolorbox}[breakable, size=fbox, boxrule=1pt, pad at break*=1mm,colback=cellbackground, colframe=cellborder]
\prompt{In}{incolor}{30}{\boxspacing}
\begin{Verbatim}[commandchars=\\\{\}]
\PY{n}{a} \PY{o}{=} \PY{n}{np}\PY{o}{.}\PY{n}{array}\PY{p}{(}\PY{p}{[}\PY{l+m+mi}{1}\PY{p}{,} \PY{l+m+mi}{2}\PY{p}{,} \PY{l+m+mi}{3}\PY{p}{,} \PY{l+m+mi}{4}\PY{p}{]}\PY{p}{)}
\PY{n}{b} \PY{o}{=} \PY{n}{np}\PY{o}{.}\PY{n}{array}\PY{p}{(}\PY{p}{[}\PY{l+m+mi}{2}\PY{p}{,} \PY{l+m+mi}{4}\PY{p}{,} \PY{l+m+mi}{6}\PY{p}{,} \PY{l+m+mi}{8}\PY{p}{]}\PY{p}{)}
\PY{n+nb}{print}\PY{p}{(}\PY{l+s+sa}{f}\PY{l+s+s2}{\PYZdq{}}\PY{l+s+s2}{a = }\PY{l+s+si}{\PYZob{}}\PY{n}{a}\PY{l+s+si}{\PYZcb{}}\PY{l+s+s2}{\PYZdq{}}\PY{p}{)}
\PY{n+nb}{print}\PY{p}{(}\PY{l+s+sa}{f}\PY{l+s+s2}{\PYZdq{}}\PY{l+s+s2}{b = }\PY{l+s+si}{\PYZob{}}\PY{n}{b}\PY{l+s+si}{\PYZcb{}}\PY{l+s+s2}{\PYZdq{}}\PY{p}{)}
\PY{n+nb}{print}\PY{p}{(}\PY{l+s+sa}{f}\PY{l+s+s2}{\PYZdq{}}\PY{l+s+s2}{b/a + 1 = }\PY{l+s+si}{\PYZob{}}\PY{n}{b}\PY{o}{/}\PY{n}{a}\PY{+w}{ }\PY{o}{+}\PY{+w}{ }\PY{l+m+mi}{1}\PY{l+s+si}{\PYZcb{}}\PY{l+s+s2}{\PYZdq{}}\PY{p}{)}
\PY{n+nb}{print}\PY{p}{(}\PY{l+s+sa}{f}\PY{l+s+s2}{\PYZdq{}}\PY{l+s+s2}{b/(a+1) = }\PY{l+s+si}{\PYZob{}}\PY{n}{b}\PY{o}{/}\PY{p}{(}\PY{n}{a}\PY{o}{+}\PY{l+m+mi}{1}\PY{p}{)}\PY{l+s+si}{\PYZcb{}}\PY{l+s+s2}{\PYZdq{}}\PY{p}{)}
\PY{n+nb}{print}\PY{p}{(}\PY{l+s+sa}{f}\PY{l+s+s2}{\PYZdq{}}\PY{l+s+s2}{1/a = }\PY{l+s+si}{\PYZob{}}\PY{l+m+mi}{1}\PY{o}{/}\PY{n}{a}\PY{l+s+si}{\PYZcb{}}\PY{l+s+s2}{\PYZdq{}}\PY{p}{)}
\end{Verbatim}
\end{tcolorbox}

    \begin{Verbatim}[commandchars=\\\{\}]
a = [1 2 3 4]
b = [2 4 6 8]
b/a + 1 = [3. 3. 3. 3.]
b/(a+1) = [1.         1.33333333 1.5        1.6       ]
1/a = [1.         0.5        0.33333333 0.25      ]
    \end{Verbatim}

    Solving for b/a + 1: b/a + 1 = {[}2/1 + 1, 4/2 + 1, 6/3 + 1, 8/4 + 1{]}
b/a + 1 = {[}2 + 1, 2 + 1, 2 + 1, 2 + 1{]} b/a + 1 = {[}3, 3, 3, 3{]}

Solving for b/(a+1): b/(a+1) = {[}2/(1+1), 4/(2+1), 6/(3+1), 8/(4+1){]}
b/(a+1) = {[}2/2, 4/3, 6/4, 8/5{]} b/(a+1) = {[}1, 1.3333, 1.5, 1.6{]}

Solving for 1/a: 1/a = {[}1/1, 1/2, 1/3, 1/4{]} 1/a = {[}1, 0.5, 0.3333,
0.25{]}

    \begin{enumerate}
\def\labelenumi{\arabic{enumi}.}
\setcounter{enumi}{1}
\tightlist
\item
  Using the same arrays above. Write a program to compute the dot
  product between them by accessing the elements of the array
  individually. Verify the result using the built-in function. (np.dot)
  You may need to look up what a dot product is.
\end{enumerate}

    \begin{tcolorbox}[breakable, size=fbox, boxrule=1pt, pad at break*=1mm,colback=cellbackground, colframe=cellborder]
\prompt{In}{incolor}{90}{\boxspacing}
\begin{Verbatim}[commandchars=\\\{\}]
\PY{k}{def} \PY{n+nf}{dotProduct}\PY{p}{(}\PY{n}{arr1}\PY{p}{,} \PY{n}{arr2}\PY{p}{)}\PY{p}{:}
    \PY{n}{product} \PY{o}{=} \PY{l+m+mi}{0}
    \PY{k}{for} \PY{n}{i} \PY{o+ow}{in} \PY{n}{np}\PY{o}{.}\PY{n}{arange}\PY{p}{(}\PY{n}{arr1}\PY{o}{.}\PY{n}{size}\PY{p}{)}\PY{p}{:}
        \PY{n+nb}{print}\PY{p}{(}\PY{l+s+sa}{f}\PY{l+s+s2}{\PYZdq{}}\PY{l+s+s2}{arr1[}\PY{l+s+si}{\PYZob{}}\PY{n}{i}\PY{l+s+si}{\PYZcb{}}\PY{l+s+s2}{] = }\PY{l+s+si}{\PYZob{}}\PY{n}{arr1}\PY{p}{[}\PY{n}{i}\PY{p}{]}\PY{l+s+si}{\PYZcb{}}\PY{l+s+se}{\PYZbs{}n}\PY{l+s+s2}{arr2[}\PY{l+s+si}{\PYZob{}}\PY{n}{i}\PY{l+s+si}{\PYZcb{}}\PY{l+s+s2}{] = }\PY{l+s+si}{\PYZob{}}\PY{n}{arr2}\PY{p}{[}\PY{n}{i}\PY{p}{]}\PY{l+s+si}{\PYZcb{}}\PY{l+s+s2}{\PYZdq{}}\PY{p}{)}
        \PY{n+nb}{print}\PY{p}{(}\PY{l+s+sa}{f}\PY{l+s+s2}{\PYZdq{}}\PY{l+s+si}{\PYZob{}}\PY{n}{arr1}\PY{p}{[}\PY{n}{i}\PY{p}{]}\PY{l+s+si}{\PYZcb{}}\PY{l+s+s2}{ * }\PY{l+s+si}{\PYZob{}}\PY{n}{arr2}\PY{p}{[}\PY{n}{i}\PY{p}{]}\PY{l+s+si}{\PYZcb{}}\PY{l+s+s2}{ = }\PY{l+s+si}{\PYZob{}}\PY{n}{arr1}\PY{p}{[}\PY{n}{i}\PY{p}{]}\PY{+w}{ }\PY{o}{*}\PY{+w}{ }\PY{n}{arr2}\PY{p}{[}\PY{n}{i}\PY{p}{]}\PY{l+s+si}{\PYZcb{}}\PY{l+s+s2}{\PYZdq{}}\PY{p}{)}
        \PY{n+nb}{print}\PY{p}{(}\PY{l+s+sa}{f}\PY{l+s+s2}{\PYZdq{}}\PY{l+s+si}{\PYZob{}}\PY{n}{product}\PY{l+s+si}{\PYZcb{}}\PY{l+s+s2}{ + }\PY{l+s+si}{\PYZob{}}\PY{n}{arr1}\PY{p}{[}\PY{n}{i}\PY{p}{]}\PY{+w}{ }\PY{o}{*}\PY{+w}{ }\PY{n}{arr2}\PY{p}{[}\PY{n}{i}\PY{p}{]}\PY{l+s+si}{\PYZcb{}}\PY{l+s+s2}{ = }\PY{l+s+si}{\PYZob{}}\PY{n}{product}\PY{+w}{ }\PY{o}{+}\PY{+w}{ }\PY{n}{arr1}\PY{p}{[}\PY{n}{i}\PY{p}{]}\PY{+w}{ }\PY{o}{*}\PY{+w}{ }\PY{n}{arr2}\PY{p}{[}\PY{n}{i}\PY{p}{]}\PY{l+s+si}{\PYZcb{}}\PY{l+s+s2}{\PYZdq{}}\PY{p}{)}
        \PY{n}{product} \PY{o}{+}\PY{o}{=} \PY{n}{arr1}\PY{p}{[}\PY{n}{i}\PY{p}{]} \PY{o}{*} \PY{n}{arr2}\PY{p}{[}\PY{n}{i}\PY{p}{]}
        \PY{n+nb}{print}\PY{p}{(}\PY{l+s+sa}{f}\PY{l+s+s2}{\PYZdq{}}\PY{l+s+s2}{dot product so far is }\PY{l+s+si}{\PYZob{}}\PY{n}{product}\PY{l+s+si}{\PYZcb{}}\PY{l+s+se}{\PYZbs{}n}\PY{l+s+s2}{\PYZdq{}}\PY{p}{)}
    \PY{k}{return} \PY{n}{product}

\PY{n}{dotProduct}\PY{p}{(}\PY{n}{a}\PY{p}{,} \PY{n}{b}\PY{p}{)}
\PY{n+nb}{print}\PY{p}{(}\PY{l+s+sa}{f}\PY{l+s+s2}{\PYZdq{}}\PY{l+s+s2}{np.dot = }\PY{l+s+si}{\PYZob{}}\PY{n}{np}\PY{o}{.}\PY{n}{dot}\PY{p}{(}\PY{n}{a}\PY{p}{,}\PY{+w}{ }\PY{n}{b}\PY{p}{)}\PY{l+s+si}{\PYZcb{}}\PY{l+s+s2}{\PYZdq{}}\PY{p}{)}
\end{Verbatim}
\end{tcolorbox}

    \begin{Verbatim}[commandchars=\\\{\}]
arr1[0] = 1
arr2[0] = 2
1 * 2 = 2
0 + 2 = 2
dot product so far is 2

arr1[1] = 2
arr2[1] = 4
2 * 4 = 8
2 + 8 = 10
dot product so far is 10

arr1[2] = 3
arr2[2] = 6
3 * 6 = 18
10 + 18 = 28
dot product so far is 28

arr1[3] = 4
arr2[3] = 8
4 * 8 = 32
28 + 32 = 60
dot product so far is 60

np.dot = 60
    \end{Verbatim}

    \begin{enumerate}
\def\labelenumi{\arabic{enumi}.}
\setcounter{enumi}{2}
\tightlist
\item
  Download the Gaussian.txt found on Canvas and put it in the same
  directory as your code. Load the values from the file using
  numpy.loadtxt() Compute the following: the sum, length, mean of all
  the values in the file. Then compute root mean square deviation (RMSD)

  \(RMSD = \sqrt{\frac{\sum_{i=0}^{i=N}(mean-values[i])^2}{N}}\)

  Since the values in the Gaussian.txt is drawn from a Gaussian
  distribution. The mean and RMSD should be almost the same as a
  Gaussian centered at 10 and with a sigma of 2.
\end{enumerate}

    \begin{tcolorbox}[breakable, size=fbox, boxrule=1pt, pad at break*=1mm,colback=cellbackground, colframe=cellborder]
\prompt{In}{incolor}{10}{\boxspacing}
\begin{Verbatim}[commandchars=\\\{\}]
\PY{n}{x} \PY{o}{=} \PY{n}{np}\PY{o}{.}\PY{n}{loadtxt}\PY{p}{(}\PY{l+s+s2}{\PYZdq{}}\PY{l+s+s2}{Gaussian.txt}\PY{l+s+s2}{\PYZdq{}}\PY{p}{)}
\PY{n+nb}{sum} \PY{o}{=} \PY{n}{np}\PY{o}{.}\PY{n}{sum}\PY{p}{(}\PY{n}{x}\PY{p}{)}
\PY{n+nb}{print}\PY{p}{(}\PY{l+s+sa}{f}\PY{l+s+s2}{\PYZdq{}}\PY{l+s+s2}{sum = }\PY{l+s+si}{\PYZob{}}\PY{n+nb}{sum}\PY{l+s+si}{\PYZcb{}}\PY{l+s+s2}{\PYZdq{}}\PY{p}{)}

\PY{n}{length} \PY{o}{=} \PY{n}{x}\PY{o}{.}\PY{n}{size}
\PY{n+nb}{print}\PY{p}{(}\PY{l+s+sa}{f}\PY{l+s+s2}{\PYZdq{}}\PY{l+s+s2}{length = }\PY{l+s+si}{\PYZob{}}\PY{n}{length}\PY{l+s+si}{\PYZcb{}}\PY{l+s+s2}{\PYZdq{}}\PY{p}{)}

\PY{n}{mean} \PY{o}{=} \PY{n}{np}\PY{o}{.}\PY{n}{mean}\PY{p}{(}\PY{n}{x}\PY{p}{)}
\PY{n+nb}{print}\PY{p}{(}\PY{l+s+sa}{f}\PY{l+s+s2}{\PYZdq{}}\PY{l+s+s2}{mean = }\PY{l+s+si}{\PYZob{}}\PY{n}{mean}\PY{l+s+si}{\PYZcb{}}\PY{l+s+s2}{\PYZdq{}}\PY{p}{)}

\PY{n}{RMSD} \PY{o}{=} \PY{n}{np}\PY{o}{.}\PY{n}{sqrt}\PY{p}{(}\PY{n}{np}\PY{o}{.}\PY{n}{sum}\PY{p}{(}\PY{p}{(}\PY{n}{x} \PY{o}{\PYZhy{}} \PY{n}{mean}\PY{p}{)}\PY{o}{*}\PY{o}{*}\PY{l+m+mi}{2}\PY{p}{)} \PY{o}{/} \PY{n}{length}\PY{p}{)}
\PY{n+nb}{print}\PY{p}{(}\PY{l+s+sa}{f}\PY{l+s+s2}{\PYZdq{}}\PY{l+s+s2}{RMSD = }\PY{l+s+si}{\PYZob{}}\PY{n}{RMSD}\PY{l+s+si}{\PYZcb{}}\PY{l+s+s2}{\PYZdq{}}\PY{p}{)}
\end{Verbatim}
\end{tcolorbox}

    \begin{Verbatim}[commandchars=\\\{\}]
sum = 10037.52371
length = 1000
mean = 10.03752371
RMSD = 1.922871400919036
    \end{Verbatim}

    \begin{enumerate}
\def\labelenumi{\arabic{enumi}.}
\setcounter{enumi}{3}
\tightlist
\item
  Download the matrix.txt found on Canvas and put in the same directory
  as your code

  \begin{itemize}
  \tightlist
  \item
    Load the values from the file using numpy.loadtxt()
  \item
    print out the following properties of the matrix and explain what
    each tell you about the array:

    \begin{itemize}
    \tightlist
    \item
      len, shape, sum, min, max (note that some of these function will
      require you loop through each row of the matrix)
    \end{itemize}
  \item
    Slice the matrix to print out only the 1st Row
  \item
    Slice the matrix to print out only the 1st Column
  \item
    Slice the matrix to print out only the 1st 3 Row
  \item
    Slice the matrix to print out only the 1st 3 Column\\
  \item
    Slice the matrix to print out only the last 3 Row
  \item
    Slice the matrix to print out only the last 3 Column
  \end{itemize}
\end{enumerate}

    \begin{tcolorbox}[breakable, size=fbox, boxrule=1pt, pad at break*=1mm,colback=cellbackground, colframe=cellborder]
\prompt{In}{incolor}{27}{\boxspacing}
\begin{Verbatim}[commandchars=\\\{\}]
\PY{n}{matrix} \PY{o}{=} \PY{n}{np}\PY{o}{.}\PY{n}{loadtxt}\PY{p}{(}\PY{l+s+s2}{\PYZdq{}}\PY{l+s+s2}{matrix.txt}\PY{l+s+s2}{\PYZdq{}}\PY{p}{)}
\PY{n+nb}{print}\PY{p}{(}\PY{l+s+sa}{f}\PY{l+s+s2}{\PYZdq{}}\PY{l+s+s2}{matrix = }\PY{l+s+se}{\PYZbs{}n}\PY{l+s+si}{\PYZob{}}\PY{n}{matrix}\PY{l+s+si}{\PYZcb{}}\PY{l+s+se}{\PYZbs{}n}\PY{l+s+s2}{\PYZdq{}}\PY{p}{)}

\PY{n+nb}{print}\PY{p}{(}\PY{l+s+sa}{f}\PY{l+s+s2}{\PYZdq{}}\PY{l+s+s2}{matrix length = }\PY{l+s+si}{\PYZob{}}\PY{n}{matrix}\PY{o}{.}\PY{n}{size}\PY{l+s+si}{\PYZcb{}}\PY{l+s+se}{\PYZbs{}n}\PY{l+s+s2}{Length describes how many elements are in the matrix.}\PY{l+s+se}{\PYZbs{}n}\PY{l+s+s2}{\PYZdq{}}\PY{p}{)}
\PY{n+nb}{print}\PY{p}{(}\PY{l+s+sa}{f}\PY{l+s+s2}{\PYZdq{}}\PY{l+s+s2}{matrix shape = }\PY{l+s+si}{\PYZob{}}\PY{n}{matrix}\PY{o}{.}\PY{n}{shape}\PY{l+s+si}{\PYZcb{}}\PY{l+s+se}{\PYZbs{}n}\PY{l+s+s2}{Shape describes the dimensions of the array.}\PY{l+s+se}{\PYZbs{}n}\PY{l+s+s2}{\PYZdq{}}\PY{p}{)}
\PY{n+nb}{print}\PY{p}{(}\PY{l+s+sa}{f}\PY{l+s+s2}{\PYZdq{}}\PY{l+s+s2}{matrix sum = }\PY{l+s+si}{\PYZob{}}\PY{n}{matrix}\PY{o}{.}\PY{n}{sum}\PY{p}{(}\PY{p}{)}\PY{l+s+si}{\PYZcb{}}\PY{l+s+se}{\PYZbs{}n}\PY{l+s+s2}{Sum describes the value when all the elements in the matrix are added together.}\PY{l+s+se}{\PYZbs{}n}\PY{l+s+s2}{\PYZdq{}}\PY{p}{)}
\PY{n+nb}{print}\PY{p}{(}\PY{l+s+sa}{f}\PY{l+s+s2}{\PYZdq{}}\PY{l+s+s2}{matrix min = }\PY{l+s+si}{\PYZob{}}\PY{n}{matrix}\PY{o}{.}\PY{n}{min}\PY{p}{(}\PY{p}{)}\PY{l+s+si}{\PYZcb{}}\PY{l+s+se}{\PYZbs{}n}\PY{l+s+s2}{Min describes the minimum value in the matrix.}\PY{l+s+se}{\PYZbs{}n}\PY{l+s+s2}{\PYZdq{}}\PY{p}{)}
\PY{n+nb}{print}\PY{p}{(}\PY{l+s+sa}{f}\PY{l+s+s2}{\PYZdq{}}\PY{l+s+s2}{matrix max = }\PY{l+s+si}{\PYZob{}}\PY{n}{matrix}\PY{o}{.}\PY{n}{max}\PY{p}{(}\PY{p}{)}\PY{l+s+si}{\PYZcb{}}\PY{l+s+se}{\PYZbs{}n}\PY{l+s+s2}{Max describes the maximum value in the matrix.}\PY{l+s+se}{\PYZbs{}n}\PY{l+s+s2}{\PYZdq{}}\PY{p}{)}
\PY{n+nb}{print}\PY{p}{(}\PY{l+s+sa}{f}\PY{l+s+s2}{\PYZdq{}}\PY{l+s+s2}{1st Row = }\PY{l+s+si}{\PYZob{}}\PY{n}{matrix}\PY{p}{[}\PY{l+m+mi}{0}\PY{p}{,}\PY{+w}{ }\PY{p}{:}\PY{p}{]}\PY{l+s+si}{\PYZcb{}}\PY{l+s+se}{\PYZbs{}n}\PY{l+s+s2}{\PYZdq{}}\PY{p}{)}
\PY{n+nb}{print}\PY{p}{(}\PY{l+s+sa}{f}\PY{l+s+s2}{\PYZdq{}}\PY{l+s+s2}{1st Column = }\PY{l+s+si}{\PYZob{}}\PY{n}{matrix}\PY{p}{[}\PY{p}{:}\PY{p}{,}\PY{+w}{ }\PY{l+m+mi}{0}\PY{p}{]}\PY{l+s+si}{\PYZcb{}}\PY{l+s+se}{\PYZbs{}n}\PY{l+s+s2}{\PYZdq{}}\PY{p}{)}
\PY{n+nb}{print}\PY{p}{(}\PY{l+s+sa}{f}\PY{l+s+s2}{\PYZdq{}}\PY{l+s+s2}{First 3 Rows =}\PY{l+s+se}{\PYZbs{}n}\PY{l+s+si}{\PYZob{}}\PY{n}{matrix}\PY{p}{[}\PY{p}{:}\PY{l+m+mi}{3}\PY{p}{,}\PY{+w}{ }\PY{p}{:}\PY{p}{]}\PY{l+s+si}{\PYZcb{}}\PY{l+s+se}{\PYZbs{}n}\PY{l+s+s2}{\PYZdq{}}\PY{p}{)}
\PY{n+nb}{print}\PY{p}{(}\PY{l+s+sa}{f}\PY{l+s+s2}{\PYZdq{}}\PY{l+s+s2}{First 3 Columns =}\PY{l+s+se}{\PYZbs{}n}\PY{l+s+si}{\PYZob{}}\PY{n}{matrix}\PY{p}{[}\PY{p}{:}\PY{p}{,}\PY{+w}{ }\PY{p}{:}\PY{l+m+mi}{3}\PY{p}{]}\PY{l+s+si}{\PYZcb{}}\PY{l+s+se}{\PYZbs{}n}\PY{l+s+s2}{\PYZdq{}}\PY{p}{)}
\PY{n+nb}{print}\PY{p}{(}\PY{l+s+sa}{f}\PY{l+s+s2}{\PYZdq{}}\PY{l+s+s2}{Last 3 Rows =}\PY{l+s+se}{\PYZbs{}n}\PY{l+s+si}{\PYZob{}}\PY{n}{matrix}\PY{p}{[}\PY{o}{\PYZhy{}}\PY{l+m+mi}{3}\PY{p}{:}\PY{p}{,}\PY{+w}{ }\PY{p}{:}\PY{p}{]}\PY{l+s+si}{\PYZcb{}}\PY{l+s+se}{\PYZbs{}n}\PY{l+s+s2}{\PYZdq{}}\PY{p}{)}
\PY{n+nb}{print}\PY{p}{(}\PY{l+s+sa}{f}\PY{l+s+s2}{\PYZdq{}}\PY{l+s+s2}{Last 3 Columns=}\PY{l+s+se}{\PYZbs{}n}\PY{l+s+si}{\PYZob{}}\PY{n}{matrix}\PY{p}{[}\PY{p}{:}\PY{p}{,}\PY{+w}{ }\PY{o}{\PYZhy{}}\PY{l+m+mi}{3}\PY{p}{:}\PY{p}{]}\PY{l+s+si}{\PYZcb{}}\PY{l+s+se}{\PYZbs{}n}\PY{l+s+s2}{\PYZdq{}}\PY{p}{)}
\end{Verbatim}
\end{tcolorbox}

    \begin{Verbatim}[commandchars=\\\{\}]
matrix =
[[ 0.  1.  2.  3.  4.  5.  6.  7.  8.  9.]
 [10. 11. 12. 13. 14. 15. 16. 17. 18. 19.]
 [20. 21. 22. 23. 24. 25. 26. 27. 28. 29.]
 [30. 31. 32. 33. 34. 35. 36. 37. 38. 39.]
 [40. 41. 42. 43. 44. 45. 46. 47. 48. 49.]
 [50. 51. 52. 53. 54. 55. 56. 57. 58. 59.]
 [60. 61. 62. 63. 64. 65. 66. 67. 68. 69.]
 [70. 71. 72. 73. 74. 75. 76. 77. 78. 79.]
 [80. 81. 82. 83. 84. 85. 86. 87. 88. 89.]
 [90. 91. 92. 93. 94. 95. 96. 97. 98. 99.]]

matrix length = 100
Length describes how many elements are in the matrix.

matrix shape = (10, 10)
Shape describes the dimensions of the array.

matrix sum = 4950.0
Sum describes the value when all the elements in the matrix are added together.

matrix min = 0.0
Min describes the minimum value in the matrix.

matrix max = 99.0
Max describes the maximum value in the matrix.

1st Row = [0. 1. 2. 3. 4. 5. 6. 7. 8. 9.]

1st Column = [ 0. 10. 20. 30. 40. 50. 60. 70. 80. 90.]

First 3 Rows =
[[ 0.  1.  2.  3.  4.  5.  6.  7.  8.  9.]
 [10. 11. 12. 13. 14. 15. 16. 17. 18. 19.]
 [20. 21. 22. 23. 24. 25. 26. 27. 28. 29.]]

First 3 Columns =
[[ 0.  1.  2.]
 [10. 11. 12.]
 [20. 21. 22.]
 [30. 31. 32.]
 [40. 41. 42.]
 [50. 51. 52.]
 [60. 61. 62.]
 [70. 71. 72.]
 [80. 81. 82.]
 [90. 91. 92.]]

Last 3 Rows =
[[70. 71. 72. 73. 74. 75. 76. 77. 78. 79.]
 [80. 81. 82. 83. 84. 85. 86. 87. 88. 89.]
 [90. 91. 92. 93. 94. 95. 96. 97. 98. 99.]]

Last 3 Columns=
[[ 7.  8.  9.]
 [17. 18. 19.]
 [27. 28. 29.]
 [37. 38. 39.]
 [47. 48. 49.]
 [57. 58. 59.]
 [67. 68. 69.]
 [77. 78. 79.]
 [87. 88. 89.]
 [97. 98. 99.]]

    \end{Verbatim}

    \begin{enumerate}
\def\labelenumi{\arabic{enumi}.}
\setcounter{enumi}{4}
\tightlist
\item
  Exercise 2.9
\end{enumerate}

    \begin{tcolorbox}[breakable, size=fbox, boxrule=1pt, pad at break*=1mm,colback=cellbackground, colframe=cellborder]
\prompt{In}{incolor}{11}{\boxspacing}
\begin{Verbatim}[commandchars=\\\{\}]
\PY{n}{L} \PY{o}{=} \PY{l+m+mi}{325}
\PY{n}{start\PYZus{}time} \PY{o}{=} \PY{n}{timeit}\PY{o}{.}\PY{n}{default\PYZus{}timer}\PY{p}{(}\PY{p}{)}

\PY{n}{i}\PY{p}{,} \PY{n}{j}\PY{p}{,} \PY{n}{k} \PY{o}{=} \PY{n}{np}\PY{o}{.}\PY{n}{meshgrid}\PY{p}{(}\PY{n}{np}\PY{o}{.}\PY{n}{arange}\PY{p}{(}\PY{o}{\PYZhy{}}\PY{n}{L}\PY{p}{,} \PY{n}{L}\PY{o}{+}\PY{l+m+mi}{1}\PY{p}{)}\PY{p}{,} \PY{n}{np}\PY{o}{.}\PY{n}{arange}\PY{p}{(}\PY{o}{\PYZhy{}}\PY{n}{L}\PY{p}{,} \PY{n}{L}\PY{o}{+}\PY{l+m+mi}{1}\PY{p}{)}\PY{p}{,} \PY{n}{np}\PY{o}{.}\PY{n}{arange}\PY{p}{(}\PY{o}{\PYZhy{}}\PY{n}{L}\PY{p}{,} \PY{n}{L}\PY{o}{+}\PY{l+m+mi}{1}\PY{p}{)}\PY{p}{,} \PY{n}{indexing}\PY{o}{=}\PY{l+s+s1}{\PYZsq{}}\PY{l+s+s1}{ij}\PY{l+s+s1}{\PYZsq{}}\PY{p}{)}

\PY{n}{r} \PY{o}{=} \PY{n}{np}\PY{o}{.}\PY{n}{sqrt}\PY{p}{(}\PY{n}{i}\PY{o}{*}\PY{o}{*}\PY{l+m+mi}{2} \PY{o}{+} \PY{n}{j}\PY{o}{*}\PY{o}{*}\PY{l+m+mi}{2} \PY{o}{+} \PY{n}{k}\PY{o}{*}\PY{o}{*}\PY{l+m+mi}{2}\PY{p}{)}
\PY{n}{r}\PY{p}{[}\PY{n}{L}\PY{p}{,} \PY{n}{L}\PY{p}{,} \PY{n}{L}\PY{p}{]} \PY{o}{=} \PY{l+m+mi}{1}

\PY{n}{sign} \PY{o}{=} \PY{n}{np}\PY{o}{.}\PY{n}{where}\PY{p}{(}\PY{p}{(}\PY{n}{i} \PY{o}{+} \PY{n}{j} \PY{o}{+} \PY{n}{k}\PY{p}{)} \PY{o}{\PYZpc{}} \PY{l+m+mi}{2} \PY{o}{==} \PY{l+m+mi}{0}\PY{p}{,} \PY{l+m+mi}{1}\PY{p}{,} \PY{o}{\PYZhy{}}\PY{l+m+mi}{1}\PY{p}{)}

\PY{n}{potentials} \PY{o}{=} \PY{n}{sign} \PY{o}{/} \PY{n}{r}

\PY{n}{total\PYZus{}sum} \PY{o}{=} \PY{n}{np}\PY{o}{.}\PY{n}{sum}\PY{p}{(}\PY{n}{potentials}\PY{p}{)} \PY{o}{\PYZhy{}} \PY{n}{potentials}\PY{p}{[}\PY{n}{L}\PY{p}{,} \PY{n}{L}\PY{p}{,} \PY{n}{L}\PY{p}{]}

\PY{n}{end\PYZus{}time} \PY{o}{=} \PY{n}{timeit}\PY{o}{.}\PY{n}{default\PYZus{}timer}\PY{p}{(}\PY{p}{)}

\PY{n}{elapsed\PYZus{}time} \PY{o}{=} \PY{n}{end\PYZus{}time} \PY{o}{\PYZhy{}} \PY{n}{start\PYZus{}time}

\PY{n+nb}{print}\PY{p}{(}\PY{l+s+sa}{f}\PY{l+s+s2}{\PYZdq{}}\PY{l+s+s2}{The Madelung constant for L = }\PY{l+s+si}{\PYZob{}}\PY{n}{L}\PY{l+s+si}{\PYZcb{}}\PY{l+s+s2}{ is approximately }\PY{l+s+si}{\PYZob{}}\PY{n}{total\PYZus{}sum}\PY{l+s+si}{\PYZcb{}}\PY{l+s+s2}{\PYZdq{}}\PY{p}{)}
\PY{n+nb}{print}\PY{p}{(}\PY{l+s+sa}{f}\PY{l+s+s2}{\PYZdq{}}\PY{l+s+s2}{Time = }\PY{l+s+si}{\PYZob{}}\PY{n}{elapsed\PYZus{}time}\PY{l+s+si}{\PYZcb{}}\PY{l+s+s2}{\PYZdq{}}\PY{p}{)}
\end{Verbatim}
\end{tcolorbox}

    \begin{Verbatim}[commandchars=\\\{\}]
The Madelung constant for L = 325 is approximately -1.7493383281793022
Time = 25.134665900026448
    \end{Verbatim}

    \begin{enumerate}
\def\labelenumi{\arabic{enumi}.}
\setcounter{enumi}{5}
\tightlist
\item
  Exercise 2.10
\end{enumerate}

    \begin{tcolorbox}[breakable, size=fbox, boxrule=1pt, pad at break*=1mm,colback=cellbackground, colframe=cellborder]
\prompt{In}{incolor}{83}{\boxspacing}
\begin{Verbatim}[commandchars=\\\{\}]
\PY{k}{def} \PY{n+nf}{calc\PYZus{}a5}\PY{p}{(}\PY{n}{A}\PY{p}{,} \PY{n}{Z}\PY{p}{)}\PY{p}{:}
    \PY{k}{if} \PY{p}{(}\PY{n}{A} \PY{o}{\PYZpc{}} \PY{l+m+mi}{2} \PY{o}{==} \PY{l+m+mi}{1}\PY{p}{)}\PY{p}{:}
        \PY{k}{return} \PY{l+m+mi}{0}
    \PY{k}{elif} \PY{p}{(}\PY{n}{A} \PY{o}{\PYZpc{}} \PY{l+m+mi}{2} \PY{o}{==} \PY{l+m+mi}{0} \PY{o+ow}{and} \PY{n}{Z} \PY{o}{\PYZpc{}} \PY{l+m+mi}{2} \PY{o}{==} \PY{l+m+mi}{0}\PY{p}{)}\PY{p}{:}
        \PY{k}{return} \PY{l+m+mi}{12}
    \PY{k}{elif} \PY{p}{(}\PY{n}{A} \PY{o}{\PYZpc{}} \PY{l+m+mi}{2} \PY{o}{==} \PY{l+m+mi}{0} \PY{o+ow}{and} \PY{n}{Z} \PY{o}{\PYZpc{}} \PY{l+m+mi}{2} \PY{o}{==} \PY{l+m+mi}{1}\PY{p}{)}\PY{p}{:}
        \PY{k}{return} \PY{o}{\PYZhy{}}\PY{l+m+mi}{12}

\PY{k}{def} \PY{n+nf}{calc\PYZus{}b}\PY{p}{(}\PY{n}{A}\PY{p}{,} \PY{n}{Z}\PY{p}{,} \PY{n}{a5}\PY{p}{)}\PY{p}{:}
    \PY{n}{a1} \PY{o}{=} \PY{l+m+mf}{15.8}
    \PY{n}{a2} \PY{o}{=} \PY{l+m+mf}{18.3}
    \PY{n}{a3} \PY{o}{=} \PY{l+m+mf}{0.714}
    \PY{n}{a4} \PY{o}{=} \PY{l+m+mf}{23.2}
    \PY{k}{return} \PY{n}{a1}\PY{o}{*}\PY{n}{A} \PY{o}{\PYZhy{}} \PY{n}{a2}\PY{o}{*}\PY{n}{A}\PY{o}{*}\PY{o}{*}\PY{p}{(}\PY{l+m+mi}{2}\PY{o}{/}\PY{l+m+mi}{3}\PY{p}{)} \PY{o}{\PYZhy{}} \PY{n}{a3}\PY{o}{*}\PY{p}{(}\PY{n}{Z}\PY{o}{*}\PY{o}{*}\PY{l+m+mi}{2}\PY{o}{/}\PY{n}{A}\PY{o}{*}\PY{o}{*}\PY{p}{(}\PY{l+m+mi}{1}\PY{o}{/}\PY{l+m+mi}{3}\PY{p}{)}\PY{p}{)} \PY{o}{\PYZhy{}} \PY{n}{a4}\PY{o}{*}\PY{p}{(}\PY{p}{(}\PY{n}{A} \PY{o}{\PYZhy{}} \PY{l+m+mi}{2}\PY{o}{*}\PY{n}{Z}\PY{p}{)}\PY{o}{*}\PY{o}{*}\PY{l+m+mi}{2}\PY{o}{/}\PY{n}{A}\PY{p}{)} \PY{o}{+} \PY{n}{a5}\PY{o}{/}\PY{n}{A}\PY{o}{*}\PY{o}{*}\PY{p}{(}\PY{l+m+mi}{1}\PY{o}{/}\PY{l+m+mi}{2}\PY{p}{)}

\PY{k}{def} \PY{n+nf}{most\PYZus{}stable\PYZus{}A}\PY{p}{(}\PY{n}{Z}\PY{p}{)}\PY{p}{:}
    \PY{n}{max\PYZus{}bepn} \PY{o}{=} \PY{o}{\PYZhy{}}\PY{l+m+mi}{1}
    \PY{n}{A\PYZus{}max} \PY{o}{=} \PY{n}{Z}

    \PY{k}{for} \PY{n}{A} \PY{o+ow}{in} \PY{n+nb}{range}\PY{p}{(}\PY{n}{Z}\PY{p}{,} \PY{l+m+mi}{3}\PY{o}{*}\PY{n}{Z} \PY{o}{+} \PY{l+m+mi}{1}\PY{p}{)}\PY{p}{:}
        \PY{n}{a5} \PY{o}{=} \PY{n}{calc\PYZus{}a5}\PY{p}{(}\PY{n}{A}\PY{p}{,} \PY{n}{Z}\PY{p}{)}
        \PY{n}{B} \PY{o}{=} \PY{n}{calc\PYZus{}b}\PY{p}{(}\PY{n}{A}\PY{p}{,} \PY{n}{Z}\PY{p}{,} \PY{n}{a5}\PY{p}{)}
        \PY{n}{bepn} \PY{o}{=} \PY{n}{B}\PY{o}{/}\PY{n}{A}

        \PY{k}{if} \PY{n}{bepn} \PY{o}{\PYZgt{}} \PY{n}{max\PYZus{}bepn}\PY{p}{:}
            \PY{n}{max\PYZus{}bepn} \PY{o}{=} \PY{n}{bepn}
            \PY{n}{A\PYZus{}max} \PY{o}{=} \PY{n}{A}

    \PY{k}{return} \PY{n}{A\PYZus{}max}\PY{p}{,} \PY{n}{max\PYZus{}bepn}

\PY{n}{A} \PY{o}{=} \PY{n+nb}{int}\PY{p}{(}\PY{n+nb}{input}\PY{p}{(}\PY{l+s+s2}{\PYZdq{}}\PY{l+s+s2}{Enter Mass Number (A):}\PY{l+s+s2}{\PYZdq{}}\PY{p}{)}\PY{p}{)}
\PY{n}{Z} \PY{o}{=} \PY{n+nb}{int}\PY{p}{(}\PY{n+nb}{input}\PY{p}{(}\PY{l+s+s2}{\PYZdq{}}\PY{l+s+s2}{Enter Atomic Number (Z):}\PY{l+s+s2}{\PYZdq{}}\PY{p}{)}\PY{p}{)}
\PY{n}{a5} \PY{o}{=} \PY{n}{calc\PYZus{}a5}\PY{p}{(}\PY{n}{A}\PY{p}{,} \PY{n}{Z}\PY{p}{)}

\PY{n}{B} \PY{o}{=} \PY{n}{calc\PYZus{}b}\PY{p}{(}\PY{n}{A}\PY{p}{,} \PY{n}{Z}\PY{p}{,} \PY{n}{a5}\PY{p}{)}
\PY{n+nb}{print}\PY{p}{(}\PY{l+s+sa}{f}\PY{l+s+s2}{\PYZdq{}}\PY{l+s+s2}{Binding Energy (B) = }\PY{l+s+si}{\PYZob{}}\PY{n}{B}\PY{l+s+si}{\PYZcb{}}\PY{l+s+s2}{\PYZdq{}}\PY{p}{)}

\PY{n}{bepn} \PY{o}{=} \PY{n}{B}\PY{o}{/}\PY{n}{A}
\PY{n+nb}{print}\PY{p}{(}\PY{l+s+sa}{f}\PY{l+s+s2}{\PYZdq{}}\PY{l+s+s2}{Binding Energy Per Nuclean = }\PY{l+s+si}{\PYZob{}}\PY{n}{bepn}\PY{l+s+si}{\PYZcb{}}\PY{l+s+s2}{\PYZdq{}}\PY{p}{)}

\PY{n}{Z} \PY{o}{=} \PY{n+nb}{int}\PY{p}{(}\PY{n+nb}{input}\PY{p}{(}\PY{l+s+s2}{\PYZdq{}}\PY{l+s+s2}{Enter Atomic Number (Z):}\PY{l+s+s2}{\PYZdq{}}\PY{p}{)}\PY{p}{)}

\PY{n}{A\PYZus{}max}\PY{p}{,} \PY{n}{max\PYZus{}bepn} \PY{o}{=} \PY{n}{most\PYZus{}stable\PYZus{}A}\PY{p}{(}\PY{n}{Z}\PY{p}{)}
\PY{n+nb}{print}\PY{p}{(}\PY{l+s+sa}{f}\PY{l+s+s2}{\PYZdq{}}\PY{l+s+s2}{The most stable nucleus has A = }\PY{l+s+si}{\PYZob{}}\PY{n}{A\PYZus{}max}\PY{l+s+si}{\PYZcb{}}\PY{l+s+s2}{\PYZdq{}}\PY{p}{)}
\PY{n+nb}{print}\PY{p}{(}\PY{l+s+sa}{f}\PY{l+s+s2}{\PYZdq{}}\PY{l+s+s2}{Binding Energy Per Nucleon = }\PY{l+s+si}{\PYZob{}}\PY{n}{max\PYZus{}bepn}\PY{l+s+si}{\PYZcb{}}\PY{l+s+s2}{\PYZdq{}}\PY{p}{)}

\PY{k}{for} \PY{n}{Z} \PY{o+ow}{in} \PY{n+nb}{range}\PY{p}{(}\PY{l+m+mi}{1}\PY{p}{,} \PY{l+m+mi}{101}\PY{p}{)}\PY{p}{:}
    \PY{n}{A\PYZus{}max}\PY{p}{,} \PY{n}{max\PYZus{}bepn} \PY{o}{=} \PY{n}{most\PYZus{}stable\PYZus{}A}\PY{p}{(}\PY{n}{Z}\PY{p}{)}    
    \PY{n+nb}{print}\PY{p}{(}\PY{l+s+sa}{f}\PY{l+s+s2}{\PYZdq{}}\PY{l+s+s2}{For Z = }\PY{l+s+si}{\PYZob{}}\PY{n}{Z}\PY{l+s+si}{\PYZcb{}}\PY{l+s+s2}{, the most stable A = }\PY{l+s+si}{\PYZob{}}\PY{n}{A\PYZus{}max}\PY{l+s+si}{\PYZcb{}}\PY{l+s+s2}{, with Binding Energy Per Nucleon = }\PY{l+s+si}{\PYZob{}}\PY{n}{max\PYZus{}bepn}\PY{l+s+si}{\PYZcb{}}\PY{l+s+s2}{\PYZdq{}}\PY{p}{)}
\end{Verbatim}
\end{tcolorbox}

    \begin{Verbatim}[commandchars=\\\{\}]
Enter Mass Number (A): 58
Enter Atomic Number (Z): 28
    \end{Verbatim}

    \begin{Verbatim}[commandchars=\\\{\}]
Binding Energy (B) = 497.5620206224374
Binding Energy Per Nuclean = 8.578655527973059
    \end{Verbatim}

    \begin{Verbatim}[commandchars=\\\{\}]
Enter Atomic Number (Z): 28
    \end{Verbatim}

    \begin{Verbatim}[commandchars=\\\{\}]
The most stable nucleus has A = 62
Binding Energy Per Nucleon = 8.70245768367189
For Z = 1, the most stable A = 3, with Binding Energy Per Nucleon =
0.36869091831015827
For Z = 2, the most stable A = 4, with Binding Energy Per Nucleon =
5.321930578649441
For Z = 3, the most stable A = 7, with Binding Energy Per Nucleon =
5.280168164356119
For Z = 4, the most stable A = 8, with Binding Energy Per Nucleon =
6.466330085889912
For Z = 5, the most stable A = 11, with Binding Energy Per Nucleon =
6.650123444727665
For Z = 6, the most stable A = 14, with Binding Energy Per Nucleon =
7.200918138809924
For Z = 7, the most stable A = 15, with Binding Energy Per Nucleon =
7.330860591990981
For Z = 8, the most stable A = 18, with Binding Energy Per Nucleon =
7.719275577459026
For Z = 9, the most stable A = 19, with Binding Energy Per Nucleon =
7.73697768275634
For Z = 10, the most stable A = 22, with Binding Energy Per Nucleon =
8.035350864715019
For Z = 11, the most stable A = 25, with Binding Energy Per Nucleon =
8.025554739665797
For Z = 12, the most stable A = 26, with Binding Energy Per Nucleon =
8.241172535624845
For Z = 13, the most stable A = 29, with Binding Energy Per Nucleon =
8.240988355754636
For Z = 14, the most stable A = 30, with Binding Energy Per Nucleon =
8.37916169002579
For Z = 15, the most stable A = 33, with Binding Energy Per Nucleon =
8.38521415855582
For Z = 16, the most stable A = 36, with Binding Energy Per Nucleon =
8.489230168218935
For Z = 17, the most stable A = 37, with Binding Energy Per Nucleon =
8.48201495174352
For Z = 18, the most stable A = 40, with Binding Energy Per Nucleon =
8.573405285254953
For Z = 19, the most stable A = 43, with Binding Energy Per Nucleon =
8.551826855569242
For Z = 20, the most stable A = 44, with Binding Energy Per Nucleon =
8.627152167121634
For Z = 21, the most stable A = 47, with Binding Energy Per Nucleon =
8.610130576802973
For Z = 22, the most stable A = 48, with Binding Energy Per Nucleon =
8.6585154571142
For Z = 23, the most stable A = 51, with Binding Energy Per Nucleon =
8.645234048730842
For Z = 24, the most stable A = 54, with Binding Energy Per Nucleon =
8.687306583887372
For Z = 25, the most stable A = 55, with Binding Energy Per Nucleon =
8.662703971015583
For Z = 26, the most stable A = 58, with Binding Energy Per Nucleon =
8.701432576808987
For Z = 27, the most stable A = 61, with Binding Energy Per Nucleon =
8.678053678353882
For Z = 28, the most stable A = 62, with Binding Energy Per Nucleon =
8.70245768367189
For Z = 29, the most stable A = 65, with Binding Energy Per Nucleon =
8.681907349580422
For Z = 30, the most stable A = 68, with Binding Energy Per Nucleon =
8.701580328486784
For Z = 31, the most stable A = 69, with Binding Energy Per Nucleon =
8.675012598311142
For Z = 32, the most stable A = 72, with Binding Energy Per Nucleon =
8.693433639739787
For Z = 33, the most stable A = 75, with Binding Energy Per Nucleon =
8.668156247337208
For Z = 34, the most stable A = 76, with Binding Energy Per Nucleon =
8.67683411103597
For Z = 35, the most stable A = 79, with Binding Energy Per Nucleon =
8.653727479263061
For Z = 36, the most stable A = 82, with Binding Energy Per Nucleon =
8.66141248935323
For Z = 37, the most stable A = 85, with Binding Energy Per Nucleon =
8.633940444065898
For Z = 38, the most stable A = 86, with Binding Energy Per Nucleon =
8.639441275530453
For Z = 39, the most stable A = 89, with Binding Energy Per Nucleon =
8.613815033672552
For Z = 40, the most stable A = 92, with Binding Energy Per Nucleon =
8.614514461544127
For Z = 41, the most stable A = 93, with Binding Energy Per Nucleon =
8.587741675710747
For Z = 42, the most stable A = 96, with Binding Energy Per Nucleon =
8.588337807352417
For Z = 43, the most stable A = 99, with Binding Energy Per Nucleon =
8.561488033970447
For Z = 44, the most stable A = 102, with Binding Energy Per Nucleon =
8.557428804696526
For Z = 45, the most stable A = 103, with Binding Energy Per Nucleon =
8.531857077904819
For Z = 46, the most stable A = 106, with Binding Energy Per Nucleon =
8.52785864674169
For Z = 47, the most stable A = 109, with Binding Energy Per Nucleon =
8.500490244095234
For Z = 48, the most stable A = 110, with Binding Energy Per Nucleon =
8.4940569666179
For Z = 49, the most stable A = 113, with Binding Energy Per Nucleon =
8.46797678166812
For Z = 50, the most stable A = 116, with Binding Energy Per Nucleon =
8.460713172533021
For Z = 51, the most stable A = 119, with Binding Energy Per Nucleon =
8.433224407343122
For Z = 52, the most stable A = 120, with Binding Energy Per Nucleon =
8.424696665334825
For Z = 53, the most stable A = 123, with Binding Energy Per Nucleon =
8.398332486638505
For Z = 54, the most stable A = 126, with Binding Energy Per Nucleon =
8.38868945344295
For Z = 55, the most stable A = 129, with Binding Energy Per Nucleon =
8.361310553455423
For Z = 56, the most stable A = 130, with Binding Energy Per Nucleon =
8.350819725669146
For Z = 57, the most stable A = 133, with Binding Energy Per Nucleon =
8.32443069811602
For Z = 58, the most stable A = 136, with Binding Energy Per Nucleon =
8.313019639868703
For Z = 59, the most stable A = 139, with Binding Energy Per Nucleon =
8.285884693887583
For Z = 60, the most stable A = 140, with Binding Energy Per Nucleon =
8.273583815729522
For Z = 61, the most stable A = 143, with Binding Energy Per Nucleon =
8.247327458286065
For Z = 62, the most stable A = 146, with Binding Energy Per Nucleon =
8.234582872513133
For Z = 63, the most stable A = 149, with Binding Energy Per Nucleon =
8.207769087134613
For Z = 64, the most stable A = 150, with Binding Energy Per Nucleon =
8.193813966434627
For Z = 65, the most stable A = 153, with Binding Energy Per Nucleon =
8.167786207290451
For Z = 66, the most stable A = 156, with Binding Energy Per Nucleon =
8.154024477520512
For Z = 67, the most stable A = 159, with Binding Energy Per Nucleon =
8.127574530301825
For Z = 68, the most stable A = 162, with Binding Energy Per Nucleon =
8.11278037767185
For Z = 69, the most stable A = 163, with Binding Energy Per Nucleon =
8.086373082934724
For Z = 70, the most stable A = 166, with Binding Energy Per Nucleon =
8.071829365612306
For Z = 71, the most stable A = 169, with Binding Energy Per Nucleon =
8.045764583108769
For Z = 72, the most stable A = 172, with Binding Energy Per Nucleon =
8.030368571593026
For Z = 73, the most stable A = 175, with Binding Energy Per Nucleon =
8.004105311543137
For Z = 74, the most stable A = 176, with Binding Energy Per Nucleon =
7.988369073415251
For Z = 75, the most stable A = 179, with Binding Energy Per Nucleon =
7.962697427019249
For Z = 76, the most stable A = 182, with Binding Energy Per Nucleon =
7.946838055684514
For Z = 77, the most stable A = 185, with Binding Energy Per Nucleon =
7.921016888554693
For Z = 78, the most stable A = 188, with Binding Energy Per Nucleon =
7.904575879934101
For Z = 79, the most stable A = 191, with Binding Energy Per Nucleon =
7.87868400232803
For Z = 80, the most stable A = 192, with Binding Energy Per Nucleon =
7.862438691993173
For Z = 81, the most stable A = 195, with Binding Energy Per Nucleon =
7.837047201045294
For Z = 82, the most stable A = 198, with Binding Energy Per Nucleon =
7.82033857608665
For Z = 83, the most stable A = 201, with Binding Energy Per Nucleon =
7.794899333942829
For Z = 84, the most stable A = 204, with Binding Energy Per Nucleon =
7.777785761025879
For Z = 85, the most stable A = 205, with Binding Energy Per Nucleon =
7.75239433360284
For Z = 86, the most stable A = 208, with Binding Energy Per Nucleon =
7.735485475322138
For Z = 87, the most stable A = 211, with Binding Energy Per Nucleon =
7.710478810144077
For Z = 88, the most stable A = 214, with Binding Energy Per Nucleon =
7.693222175964613
For Z = 89, the most stable A = 217, with Binding Energy Per Nucleon =
7.668228396728228
For Z = 90, the most stable A = 220, with Binding Energy Per Nucleon =
7.65068598823387
For Z = 91, the most stable A = 223, with Binding Energy Per Nucleon =
7.625739835440575
For Z = 92, the most stable A = 224, with Binding Energy Per Nucleon =
7.608214013689897
For Z = 93, the most stable A = 227, with Binding Energy Per Nucleon =
7.583639834526779
For Z = 94, the most stable A = 230, with Binding Energy Per Nucleon =
7.566035830526522
For Z = 95, the most stable A = 233, with Binding Energy Per Nucleon =
7.5415108314640555
For Z = 96, the most stable A = 236, with Binding Energy Per Nucleon =
7.523703637516345
For Z = 97, the most stable A = 239, with Binding Energy Per Nucleon =
7.499251800171257
For Z = 98, the most stable A = 242, with Binding Energy Per Nucleon =
7.481279349508352
For Z = 99, the most stable A = 243, with Binding Energy Per Nucleon =
7.456937323389022
For Z = 100, the most stable A = 246, with Binding Energy Per Nucleon =
7.439122944429214
    \end{Verbatim}

    \begin{tcolorbox}[breakable, size=fbox, boxrule=1pt, pad at break*=1mm,colback=cellbackground, colframe=cellborder]
\prompt{In}{incolor}{ }{\boxspacing}
\begin{Verbatim}[commandchars=\\\{\}]

\end{Verbatim}
\end{tcolorbox}


    % Add a bibliography block to the postdoc
    
    
    
\end{document}
