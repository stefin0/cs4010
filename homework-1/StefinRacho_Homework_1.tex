\documentclass[11pt]{article}

    \usepackage[breakable]{tcolorbox}
    \usepackage{parskip} % Stop auto-indenting (to mimic markdown behaviour)
    

    % Basic figure setup, for now with no caption control since it's done
    % automatically by Pandoc (which extracts ![](path) syntax from Markdown).
    \usepackage{graphicx}
    % Keep aspect ratio if custom image width or height is specified
    \setkeys{Gin}{keepaspectratio}
    % Maintain compatibility with old templates. Remove in nbconvert 6.0
    \let\Oldincludegraphics\includegraphics
    % Ensure that by default, figures have no caption (until we provide a
    % proper Figure object with a Caption API and a way to capture that
    % in the conversion process - todo).
    \usepackage{caption}
    \DeclareCaptionFormat{nocaption}{}
    \captionsetup{format=nocaption,aboveskip=0pt,belowskip=0pt}

    \usepackage{float}
    \floatplacement{figure}{H} % forces figures to be placed at the correct location
    \usepackage{xcolor} % Allow colors to be defined
    \usepackage{enumerate} % Needed for markdown enumerations to work
    \usepackage{geometry} % Used to adjust the document margins
    \usepackage{amsmath} % Equations
    \usepackage{amssymb} % Equations
    \usepackage{textcomp} % defines textquotesingle
    % Hack from http://tex.stackexchange.com/a/47451/13684:
    \AtBeginDocument{%
        \def\PYZsq{\textquotesingle}% Upright quotes in Pygmentized code
    }
    \usepackage{upquote} % Upright quotes for verbatim code
    \usepackage{eurosym} % defines \euro

    \usepackage{iftex}
    \ifPDFTeX
        \usepackage[T1]{fontenc}
        \IfFileExists{alphabeta.sty}{
              \usepackage{alphabeta}
          }{
              \usepackage[mathletters]{ucs}
              \usepackage[utf8x]{inputenc}
          }
    \else
        \usepackage{fontspec}
        \usepackage{unicode-math}
    \fi

    \usepackage{fancyvrb} % verbatim replacement that allows latex
    \usepackage{grffile} % extends the file name processing of package graphics
                         % to support a larger range
    \makeatletter % fix for old versions of grffile with XeLaTeX
    \@ifpackagelater{grffile}{2019/11/01}
    {
      % Do nothing on new versions
    }
    {
      \def\Gread@@xetex#1{%
        \IfFileExists{"\Gin@base".bb}%
        {\Gread@eps{\Gin@base.bb}}%
        {\Gread@@xetex@aux#1}%
      }
    }
    \makeatother
    \usepackage[Export]{adjustbox} % Used to constrain images to a maximum size
    \adjustboxset{max size={0.9\linewidth}{0.9\paperheight}}

    % The hyperref package gives us a pdf with properly built
    % internal navigation ('pdf bookmarks' for the table of contents,
    % internal cross-reference links, web links for URLs, etc.)
    \usepackage{hyperref}
    % The default LaTeX title has an obnoxious amount of whitespace. By default,
    % titling removes some of it. It also provides customization options.
    \usepackage{titling}
    \usepackage{longtable} % longtable support required by pandoc >1.10
    \usepackage{booktabs}  % table support for pandoc > 1.12.2
    \usepackage{array}     % table support for pandoc >= 2.11.3
    \usepackage{calc}      % table minipage width calculation for pandoc >= 2.11.1
    \usepackage[inline]{enumitem} % IRkernel/repr support (it uses the enumerate* environment)
    \usepackage[normalem]{ulem} % ulem is needed to support strikethroughs (\sout)
                                % normalem makes italics be italics, not underlines
    \usepackage{soul}      % strikethrough (\st) support for pandoc >= 3.0.0
    \usepackage{mathrsfs}
    

    
    % Colors for the hyperref package
    \definecolor{urlcolor}{rgb}{0,.145,.698}
    \definecolor{linkcolor}{rgb}{.71,0.21,0.01}
    \definecolor{citecolor}{rgb}{.12,.54,.11}

    % ANSI colors
    \definecolor{ansi-black}{HTML}{3E424D}
    \definecolor{ansi-black-intense}{HTML}{282C36}
    \definecolor{ansi-red}{HTML}{E75C58}
    \definecolor{ansi-red-intense}{HTML}{B22B31}
    \definecolor{ansi-green}{HTML}{00A250}
    \definecolor{ansi-green-intense}{HTML}{007427}
    \definecolor{ansi-yellow}{HTML}{DDB62B}
    \definecolor{ansi-yellow-intense}{HTML}{B27D12}
    \definecolor{ansi-blue}{HTML}{208FFB}
    \definecolor{ansi-blue-intense}{HTML}{0065CA}
    \definecolor{ansi-magenta}{HTML}{D160C4}
    \definecolor{ansi-magenta-intense}{HTML}{A03196}
    \definecolor{ansi-cyan}{HTML}{60C6C8}
    \definecolor{ansi-cyan-intense}{HTML}{258F8F}
    \definecolor{ansi-white}{HTML}{C5C1B4}
    \definecolor{ansi-white-intense}{HTML}{A1A6B2}
    \definecolor{ansi-default-inverse-fg}{HTML}{FFFFFF}
    \definecolor{ansi-default-inverse-bg}{HTML}{000000}

    % common color for the border for error outputs.
    \definecolor{outerrorbackground}{HTML}{FFDFDF}

    % commands and environments needed by pandoc snippets
    % extracted from the output of `pandoc -s`
    \providecommand{\tightlist}{%
      \setlength{\itemsep}{0pt}\setlength{\parskip}{0pt}}
    \DefineVerbatimEnvironment{Highlighting}{Verbatim}{commandchars=\\\{\}}
    % Add ',fontsize=\small' for more characters per line
    \newenvironment{Shaded}{}{}
    \newcommand{\KeywordTok}[1]{\textcolor[rgb]{0.00,0.44,0.13}{\textbf{{#1}}}}
    \newcommand{\DataTypeTok}[1]{\textcolor[rgb]{0.56,0.13,0.00}{{#1}}}
    \newcommand{\DecValTok}[1]{\textcolor[rgb]{0.25,0.63,0.44}{{#1}}}
    \newcommand{\BaseNTok}[1]{\textcolor[rgb]{0.25,0.63,0.44}{{#1}}}
    \newcommand{\FloatTok}[1]{\textcolor[rgb]{0.25,0.63,0.44}{{#1}}}
    \newcommand{\CharTok}[1]{\textcolor[rgb]{0.25,0.44,0.63}{{#1}}}
    \newcommand{\StringTok}[1]{\textcolor[rgb]{0.25,0.44,0.63}{{#1}}}
    \newcommand{\CommentTok}[1]{\textcolor[rgb]{0.38,0.63,0.69}{\textit{{#1}}}}
    \newcommand{\OtherTok}[1]{\textcolor[rgb]{0.00,0.44,0.13}{{#1}}}
    \newcommand{\AlertTok}[1]{\textcolor[rgb]{1.00,0.00,0.00}{\textbf{{#1}}}}
    \newcommand{\FunctionTok}[1]{\textcolor[rgb]{0.02,0.16,0.49}{{#1}}}
    \newcommand{\RegionMarkerTok}[1]{{#1}}
    \newcommand{\ErrorTok}[1]{\textcolor[rgb]{1.00,0.00,0.00}{\textbf{{#1}}}}
    \newcommand{\NormalTok}[1]{{#1}}

    % Additional commands for more recent versions of Pandoc
    \newcommand{\ConstantTok}[1]{\textcolor[rgb]{0.53,0.00,0.00}{{#1}}}
    \newcommand{\SpecialCharTok}[1]{\textcolor[rgb]{0.25,0.44,0.63}{{#1}}}
    \newcommand{\VerbatimStringTok}[1]{\textcolor[rgb]{0.25,0.44,0.63}{{#1}}}
    \newcommand{\SpecialStringTok}[1]{\textcolor[rgb]{0.73,0.40,0.53}{{#1}}}
    \newcommand{\ImportTok}[1]{{#1}}
    \newcommand{\DocumentationTok}[1]{\textcolor[rgb]{0.73,0.13,0.13}{\textit{{#1}}}}
    \newcommand{\AnnotationTok}[1]{\textcolor[rgb]{0.38,0.63,0.69}{\textbf{\textit{{#1}}}}}
    \newcommand{\CommentVarTok}[1]{\textcolor[rgb]{0.38,0.63,0.69}{\textbf{\textit{{#1}}}}}
    \newcommand{\VariableTok}[1]{\textcolor[rgb]{0.10,0.09,0.49}{{#1}}}
    \newcommand{\ControlFlowTok}[1]{\textcolor[rgb]{0.00,0.44,0.13}{\textbf{{#1}}}}
    \newcommand{\OperatorTok}[1]{\textcolor[rgb]{0.40,0.40,0.40}{{#1}}}
    \newcommand{\BuiltInTok}[1]{{#1}}
    \newcommand{\ExtensionTok}[1]{{#1}}
    \newcommand{\PreprocessorTok}[1]{\textcolor[rgb]{0.74,0.48,0.00}{{#1}}}
    \newcommand{\AttributeTok}[1]{\textcolor[rgb]{0.49,0.56,0.16}{{#1}}}
    \newcommand{\InformationTok}[1]{\textcolor[rgb]{0.38,0.63,0.69}{\textbf{\textit{{#1}}}}}
    \newcommand{\WarningTok}[1]{\textcolor[rgb]{0.38,0.63,0.69}{\textbf{\textit{{#1}}}}}


    % Define a nice break command that doesn't care if a line doesn't already
    % exist.
    \def\br{\hspace*{\fill} \\* }
    % Math Jax compatibility definitions
    \def\gt{>}
    \def\lt{<}
    \let\Oldtex\TeX
    \let\Oldlatex\LaTeX
    \renewcommand{\TeX}{\textrm{\Oldtex}}
    \renewcommand{\LaTeX}{\textrm{\Oldlatex}}
    % Document parameters
    % Document title
    \title{StefinRacho\_Homework\_1}
    
    
    
    
    
    
    
% Pygments definitions
\makeatletter
\def\PY@reset{\let\PY@it=\relax \let\PY@bf=\relax%
    \let\PY@ul=\relax \let\PY@tc=\relax%
    \let\PY@bc=\relax \let\PY@ff=\relax}
\def\PY@tok#1{\csname PY@tok@#1\endcsname}
\def\PY@toks#1+{\ifx\relax#1\empty\else%
    \PY@tok{#1}\expandafter\PY@toks\fi}
\def\PY@do#1{\PY@bc{\PY@tc{\PY@ul{%
    \PY@it{\PY@bf{\PY@ff{#1}}}}}}}
\def\PY#1#2{\PY@reset\PY@toks#1+\relax+\PY@do{#2}}

\@namedef{PY@tok@w}{\def\PY@tc##1{\textcolor[rgb]{0.73,0.73,0.73}{##1}}}
\@namedef{PY@tok@c}{\let\PY@it=\textit\def\PY@tc##1{\textcolor[rgb]{0.24,0.48,0.48}{##1}}}
\@namedef{PY@tok@cp}{\def\PY@tc##1{\textcolor[rgb]{0.61,0.40,0.00}{##1}}}
\@namedef{PY@tok@k}{\let\PY@bf=\textbf\def\PY@tc##1{\textcolor[rgb]{0.00,0.50,0.00}{##1}}}
\@namedef{PY@tok@kp}{\def\PY@tc##1{\textcolor[rgb]{0.00,0.50,0.00}{##1}}}
\@namedef{PY@tok@kt}{\def\PY@tc##1{\textcolor[rgb]{0.69,0.00,0.25}{##1}}}
\@namedef{PY@tok@o}{\def\PY@tc##1{\textcolor[rgb]{0.40,0.40,0.40}{##1}}}
\@namedef{PY@tok@ow}{\let\PY@bf=\textbf\def\PY@tc##1{\textcolor[rgb]{0.67,0.13,1.00}{##1}}}
\@namedef{PY@tok@nb}{\def\PY@tc##1{\textcolor[rgb]{0.00,0.50,0.00}{##1}}}
\@namedef{PY@tok@nf}{\def\PY@tc##1{\textcolor[rgb]{0.00,0.00,1.00}{##1}}}
\@namedef{PY@tok@nc}{\let\PY@bf=\textbf\def\PY@tc##1{\textcolor[rgb]{0.00,0.00,1.00}{##1}}}
\@namedef{PY@tok@nn}{\let\PY@bf=\textbf\def\PY@tc##1{\textcolor[rgb]{0.00,0.00,1.00}{##1}}}
\@namedef{PY@tok@ne}{\let\PY@bf=\textbf\def\PY@tc##1{\textcolor[rgb]{0.80,0.25,0.22}{##1}}}
\@namedef{PY@tok@nv}{\def\PY@tc##1{\textcolor[rgb]{0.10,0.09,0.49}{##1}}}
\@namedef{PY@tok@no}{\def\PY@tc##1{\textcolor[rgb]{0.53,0.00,0.00}{##1}}}
\@namedef{PY@tok@nl}{\def\PY@tc##1{\textcolor[rgb]{0.46,0.46,0.00}{##1}}}
\@namedef{PY@tok@ni}{\let\PY@bf=\textbf\def\PY@tc##1{\textcolor[rgb]{0.44,0.44,0.44}{##1}}}
\@namedef{PY@tok@na}{\def\PY@tc##1{\textcolor[rgb]{0.41,0.47,0.13}{##1}}}
\@namedef{PY@tok@nt}{\let\PY@bf=\textbf\def\PY@tc##1{\textcolor[rgb]{0.00,0.50,0.00}{##1}}}
\@namedef{PY@tok@nd}{\def\PY@tc##1{\textcolor[rgb]{0.67,0.13,1.00}{##1}}}
\@namedef{PY@tok@s}{\def\PY@tc##1{\textcolor[rgb]{0.73,0.13,0.13}{##1}}}
\@namedef{PY@tok@sd}{\let\PY@it=\textit\def\PY@tc##1{\textcolor[rgb]{0.73,0.13,0.13}{##1}}}
\@namedef{PY@tok@si}{\let\PY@bf=\textbf\def\PY@tc##1{\textcolor[rgb]{0.64,0.35,0.47}{##1}}}
\@namedef{PY@tok@se}{\let\PY@bf=\textbf\def\PY@tc##1{\textcolor[rgb]{0.67,0.36,0.12}{##1}}}
\@namedef{PY@tok@sr}{\def\PY@tc##1{\textcolor[rgb]{0.64,0.35,0.47}{##1}}}
\@namedef{PY@tok@ss}{\def\PY@tc##1{\textcolor[rgb]{0.10,0.09,0.49}{##1}}}
\@namedef{PY@tok@sx}{\def\PY@tc##1{\textcolor[rgb]{0.00,0.50,0.00}{##1}}}
\@namedef{PY@tok@m}{\def\PY@tc##1{\textcolor[rgb]{0.40,0.40,0.40}{##1}}}
\@namedef{PY@tok@gh}{\let\PY@bf=\textbf\def\PY@tc##1{\textcolor[rgb]{0.00,0.00,0.50}{##1}}}
\@namedef{PY@tok@gu}{\let\PY@bf=\textbf\def\PY@tc##1{\textcolor[rgb]{0.50,0.00,0.50}{##1}}}
\@namedef{PY@tok@gd}{\def\PY@tc##1{\textcolor[rgb]{0.63,0.00,0.00}{##1}}}
\@namedef{PY@tok@gi}{\def\PY@tc##1{\textcolor[rgb]{0.00,0.52,0.00}{##1}}}
\@namedef{PY@tok@gr}{\def\PY@tc##1{\textcolor[rgb]{0.89,0.00,0.00}{##1}}}
\@namedef{PY@tok@ge}{\let\PY@it=\textit}
\@namedef{PY@tok@gs}{\let\PY@bf=\textbf}
\@namedef{PY@tok@ges}{\let\PY@bf=\textbf\let\PY@it=\textit}
\@namedef{PY@tok@gp}{\let\PY@bf=\textbf\def\PY@tc##1{\textcolor[rgb]{0.00,0.00,0.50}{##1}}}
\@namedef{PY@tok@go}{\def\PY@tc##1{\textcolor[rgb]{0.44,0.44,0.44}{##1}}}
\@namedef{PY@tok@gt}{\def\PY@tc##1{\textcolor[rgb]{0.00,0.27,0.87}{##1}}}
\@namedef{PY@tok@err}{\def\PY@bc##1{{\setlength{\fboxsep}{\string -\fboxrule}\fcolorbox[rgb]{1.00,0.00,0.00}{1,1,1}{\strut ##1}}}}
\@namedef{PY@tok@kc}{\let\PY@bf=\textbf\def\PY@tc##1{\textcolor[rgb]{0.00,0.50,0.00}{##1}}}
\@namedef{PY@tok@kd}{\let\PY@bf=\textbf\def\PY@tc##1{\textcolor[rgb]{0.00,0.50,0.00}{##1}}}
\@namedef{PY@tok@kn}{\let\PY@bf=\textbf\def\PY@tc##1{\textcolor[rgb]{0.00,0.50,0.00}{##1}}}
\@namedef{PY@tok@kr}{\let\PY@bf=\textbf\def\PY@tc##1{\textcolor[rgb]{0.00,0.50,0.00}{##1}}}
\@namedef{PY@tok@bp}{\def\PY@tc##1{\textcolor[rgb]{0.00,0.50,0.00}{##1}}}
\@namedef{PY@tok@fm}{\def\PY@tc##1{\textcolor[rgb]{0.00,0.00,1.00}{##1}}}
\@namedef{PY@tok@vc}{\def\PY@tc##1{\textcolor[rgb]{0.10,0.09,0.49}{##1}}}
\@namedef{PY@tok@vg}{\def\PY@tc##1{\textcolor[rgb]{0.10,0.09,0.49}{##1}}}
\@namedef{PY@tok@vi}{\def\PY@tc##1{\textcolor[rgb]{0.10,0.09,0.49}{##1}}}
\@namedef{PY@tok@vm}{\def\PY@tc##1{\textcolor[rgb]{0.10,0.09,0.49}{##1}}}
\@namedef{PY@tok@sa}{\def\PY@tc##1{\textcolor[rgb]{0.73,0.13,0.13}{##1}}}
\@namedef{PY@tok@sb}{\def\PY@tc##1{\textcolor[rgb]{0.73,0.13,0.13}{##1}}}
\@namedef{PY@tok@sc}{\def\PY@tc##1{\textcolor[rgb]{0.73,0.13,0.13}{##1}}}
\@namedef{PY@tok@dl}{\def\PY@tc##1{\textcolor[rgb]{0.73,0.13,0.13}{##1}}}
\@namedef{PY@tok@s2}{\def\PY@tc##1{\textcolor[rgb]{0.73,0.13,0.13}{##1}}}
\@namedef{PY@tok@sh}{\def\PY@tc##1{\textcolor[rgb]{0.73,0.13,0.13}{##1}}}
\@namedef{PY@tok@s1}{\def\PY@tc##1{\textcolor[rgb]{0.73,0.13,0.13}{##1}}}
\@namedef{PY@tok@mb}{\def\PY@tc##1{\textcolor[rgb]{0.40,0.40,0.40}{##1}}}
\@namedef{PY@tok@mf}{\def\PY@tc##1{\textcolor[rgb]{0.40,0.40,0.40}{##1}}}
\@namedef{PY@tok@mh}{\def\PY@tc##1{\textcolor[rgb]{0.40,0.40,0.40}{##1}}}
\@namedef{PY@tok@mi}{\def\PY@tc##1{\textcolor[rgb]{0.40,0.40,0.40}{##1}}}
\@namedef{PY@tok@il}{\def\PY@tc##1{\textcolor[rgb]{0.40,0.40,0.40}{##1}}}
\@namedef{PY@tok@mo}{\def\PY@tc##1{\textcolor[rgb]{0.40,0.40,0.40}{##1}}}
\@namedef{PY@tok@ch}{\let\PY@it=\textit\def\PY@tc##1{\textcolor[rgb]{0.24,0.48,0.48}{##1}}}
\@namedef{PY@tok@cm}{\let\PY@it=\textit\def\PY@tc##1{\textcolor[rgb]{0.24,0.48,0.48}{##1}}}
\@namedef{PY@tok@cpf}{\let\PY@it=\textit\def\PY@tc##1{\textcolor[rgb]{0.24,0.48,0.48}{##1}}}
\@namedef{PY@tok@c1}{\let\PY@it=\textit\def\PY@tc##1{\textcolor[rgb]{0.24,0.48,0.48}{##1}}}
\@namedef{PY@tok@cs}{\let\PY@it=\textit\def\PY@tc##1{\textcolor[rgb]{0.24,0.48,0.48}{##1}}}

\def\PYZbs{\char`\\}
\def\PYZus{\char`\_}
\def\PYZob{\char`\{}
\def\PYZcb{\char`\}}
\def\PYZca{\char`\^}
\def\PYZam{\char`\&}
\def\PYZlt{\char`\<}
\def\PYZgt{\char`\>}
\def\PYZsh{\char`\#}
\def\PYZpc{\char`\%}
\def\PYZdl{\char`\$}
\def\PYZhy{\char`\-}
\def\PYZsq{\char`\'}
\def\PYZdq{\char`\"}
\def\PYZti{\char`\~}
% for compatibility with earlier versions
\def\PYZat{@}
\def\PYZlb{[}
\def\PYZrb{]}
\makeatother


    % For linebreaks inside Verbatim environment from package fancyvrb.
    \makeatletter
        \newbox\Wrappedcontinuationbox
        \newbox\Wrappedvisiblespacebox
        \newcommand*\Wrappedvisiblespace {\textcolor{red}{\textvisiblespace}}
        \newcommand*\Wrappedcontinuationsymbol {\textcolor{red}{\llap{\tiny$\m@th\hookrightarrow$}}}
        \newcommand*\Wrappedcontinuationindent {3ex }
        \newcommand*\Wrappedafterbreak {\kern\Wrappedcontinuationindent\copy\Wrappedcontinuationbox}
        % Take advantage of the already applied Pygments mark-up to insert
        % potential linebreaks for TeX processing.
        %        {, <, #, %, $, ' and ": go to next line.
        %        _, }, ^, &, >, - and ~: stay at end of broken line.
        % Use of \textquotesingle for straight quote.
        \newcommand*\Wrappedbreaksatspecials {%
            \def\PYGZus{\discretionary{\char`\_}{\Wrappedafterbreak}{\char`\_}}%
            \def\PYGZob{\discretionary{}{\Wrappedafterbreak\char`\{}{\char`\{}}%
            \def\PYGZcb{\discretionary{\char`\}}{\Wrappedafterbreak}{\char`\}}}%
            \def\PYGZca{\discretionary{\char`\^}{\Wrappedafterbreak}{\char`\^}}%
            \def\PYGZam{\discretionary{\char`\&}{\Wrappedafterbreak}{\char`\&}}%
            \def\PYGZlt{\discretionary{}{\Wrappedafterbreak\char`\<}{\char`\<}}%
            \def\PYGZgt{\discretionary{\char`\>}{\Wrappedafterbreak}{\char`\>}}%
            \def\PYGZsh{\discretionary{}{\Wrappedafterbreak\char`\#}{\char`\#}}%
            \def\PYGZpc{\discretionary{}{\Wrappedafterbreak\char`\%}{\char`\%}}%
            \def\PYGZdl{\discretionary{}{\Wrappedafterbreak\char`\$}{\char`\$}}%
            \def\PYGZhy{\discretionary{\char`\-}{\Wrappedafterbreak}{\char`\-}}%
            \def\PYGZsq{\discretionary{}{\Wrappedafterbreak\textquotesingle}{\textquotesingle}}%
            \def\PYGZdq{\discretionary{}{\Wrappedafterbreak\char`\"}{\char`\"}}%
            \def\PYGZti{\discretionary{\char`\~}{\Wrappedafterbreak}{\char`\~}}%
        }
        % Some characters . , ; ? ! / are not pygmentized.
        % This macro makes them "active" and they will insert potential linebreaks
        \newcommand*\Wrappedbreaksatpunct {%
            \lccode`\~`\.\lowercase{\def~}{\discretionary{\hbox{\char`\.}}{\Wrappedafterbreak}{\hbox{\char`\.}}}%
            \lccode`\~`\,\lowercase{\def~}{\discretionary{\hbox{\char`\,}}{\Wrappedafterbreak}{\hbox{\char`\,}}}%
            \lccode`\~`\;\lowercase{\def~}{\discretionary{\hbox{\char`\;}}{\Wrappedafterbreak}{\hbox{\char`\;}}}%
            \lccode`\~`\:\lowercase{\def~}{\discretionary{\hbox{\char`\:}}{\Wrappedafterbreak}{\hbox{\char`\:}}}%
            \lccode`\~`\?\lowercase{\def~}{\discretionary{\hbox{\char`\?}}{\Wrappedafterbreak}{\hbox{\char`\?}}}%
            \lccode`\~`\!\lowercase{\def~}{\discretionary{\hbox{\char`\!}}{\Wrappedafterbreak}{\hbox{\char`\!}}}%
            \lccode`\~`\/\lowercase{\def~}{\discretionary{\hbox{\char`\/}}{\Wrappedafterbreak}{\hbox{\char`\/}}}%
            \catcode`\.\active
            \catcode`\,\active
            \catcode`\;\active
            \catcode`\:\active
            \catcode`\?\active
            \catcode`\!\active
            \catcode`\/\active
            \lccode`\~`\~
        }
    \makeatother

    \let\OriginalVerbatim=\Verbatim
    \makeatletter
    \renewcommand{\Verbatim}[1][1]{%
        %\parskip\z@skip
        \sbox\Wrappedcontinuationbox {\Wrappedcontinuationsymbol}%
        \sbox\Wrappedvisiblespacebox {\FV@SetupFont\Wrappedvisiblespace}%
        \def\FancyVerbFormatLine ##1{\hsize\linewidth
            \vtop{\raggedright\hyphenpenalty\z@\exhyphenpenalty\z@
                \doublehyphendemerits\z@\finalhyphendemerits\z@
                \strut ##1\strut}%
        }%
        % If the linebreak is at a space, the latter will be displayed as visible
        % space at end of first line, and a continuation symbol starts next line.
        % Stretch/shrink are however usually zero for typewriter font.
        \def\FV@Space {%
            \nobreak\hskip\z@ plus\fontdimen3\font minus\fontdimen4\font
            \discretionary{\copy\Wrappedvisiblespacebox}{\Wrappedafterbreak}
            {\kern\fontdimen2\font}%
        }%

        % Allow breaks at special characters using \PYG... macros.
        \Wrappedbreaksatspecials
        % Breaks at punctuation characters . , ; ? ! and / need catcode=\active
        \OriginalVerbatim[#1,codes*=\Wrappedbreaksatpunct]%
    }
    \makeatother

    % Exact colors from NB
    \definecolor{incolor}{HTML}{303F9F}
    \definecolor{outcolor}{HTML}{D84315}
    \definecolor{cellborder}{HTML}{CFCFCF}
    \definecolor{cellbackground}{HTML}{F7F7F7}

    % prompt
    \makeatletter
    \newcommand{\boxspacing}{\kern\kvtcb@left@rule\kern\kvtcb@boxsep}
    \makeatother
    \newcommand{\prompt}[4]{
        {\ttfamily\llap{{\color{#2}[#3]:\hspace{3pt}#4}}\vspace{-\baselineskip}}
    }
    

    
    % Prevent overflowing lines due to hard-to-break entities
    \sloppy
    % Setup hyperref package
    \hypersetup{
      breaklinks=true,  % so long urls are correctly broken across lines
      colorlinks=true,
      urlcolor=urlcolor,
      linkcolor=linkcolor,
      citecolor=citecolor,
      }
    % Slightly bigger margins than the latex defaults
    
    \geometry{verbose,tmargin=1in,bmargin=1in,lmargin=1in,rmargin=1in}
    
    

\begin{document}
    
    \maketitle
    
    

    
    CS 4010 - Homework 1

Due Friday, 9/13/2024

For each homework set: Create a New iPython (Jupyter) notebook. Name the
notebook FirstAndLastName\_Homework1 and save it. To submit, export or
print your notebook as a pdf, with all outputs visible. Upload both the
pdf and a copy of your notebook (.ipynb) in Canvas.

\begin{enumerate}
\def\labelenumi{\arabic{enumi}.}
\tightlist
\item
  For a computer with 4 GB of RAM, how many int values can the RAM hold?
  Note that your computer might have a different size of int. Make sure
  your code checks this and compute it properly.

  \begin{itemize}
  \tightlist
  \item
    Repeat for Complex values.
  \end{itemize}
\end{enumerate}

    \begin{tcolorbox}[breakable, size=fbox, boxrule=1pt, pad at break*=1mm,colback=cellbackground, colframe=cellborder]
\prompt{In}{incolor}{2}{\boxspacing}
\begin{Verbatim}[commandchars=\\\{\}]
\PY{c+c1}{\PYZsh{} The following code was made with knowledge from}
\PY{c+c1}{\PYZsh{} the following link: https://www.geeksforgeeks.org/how\PYZhy{}to\PYZhy{}find\PYZhy{}size\PYZhy{}of\PYZhy{}an\PYZhy{}object\PYZhy{}in\PYZhy{}python/}
\PY{k+kn}{import} \PY{n+nn}{sys}

\PY{n}{int\PYZus{}size} \PY{o}{=} \PY{n}{sys}\PY{o}{.}\PY{n}{getsizeof}\PY{p}{(}\PY{l+m+mi}{0}\PY{p}{)}
\PY{n+nb}{print}\PY{p}{(}\PY{l+s+sa}{f}\PY{l+s+s2}{\PYZdq{}}\PY{l+s+s2}{The size of an int in python (on this computer) is: }\PY{l+s+si}{\PYZob{}}\PY{n}{int\PYZus{}size}\PY{l+s+si}{\PYZcb{}}\PY{l+s+s2}{ bytes.}\PY{l+s+s2}{\PYZdq{}}\PY{p}{)}

\PY{c+c1}{\PYZsh{} To calculate how many int values the }
\PY{c+c1}{\PYZsh{} RAM can hold we need to divide 4 billion bytes by}
\PY{c+c1}{\PYZsh{} the size of an int.}
\PY{n}{ram\PYZus{}size} \PY{o}{=} \PY{l+m+mi}{4} \PY{o}{*} \PY{p}{(}\PY{l+m+mi}{10}\PY{o}{*}\PY{o}{*}\PY{l+m+mi}{9}\PY{p}{)} \PY{c+c1}{\PYZsh{} 4 GB in bytes}
\PY{n}{num\PYZus{}ints} \PY{o}{=} \PY{n}{ram\PYZus{}size} \PY{o}{/}\PY{o}{/} \PY{n}{int\PYZus{}size}

\PY{n+nb}{print}\PY{p}{(}\PY{l+s+sa}{f}\PY{l+s+s2}{\PYZdq{}}\PY{l+s+s2}{This computer with 4 GB of RAM can hold approximately }\PY{l+s+si}{\PYZob{}}\PY{n}{num\PYZus{}ints}\PY{l+s+si}{\PYZcb{}}\PY{l+s+s2}{ int values.}\PY{l+s+s2}{\PYZdq{}}\PY{p}{)}
\end{Verbatim}
\end{tcolorbox}

    \begin{Verbatim}[commandchars=\\\{\}]
The size of an int in python (on this computer) is: 28 bytes.
This computer with 4 GB of RAM can hold approximately 142857142 int values.
    \end{Verbatim}

    \begin{enumerate}
\def\labelenumi{\arabic{enumi}.}
\setcounter{enumi}{1}
\tightlist
\item
  Search for Mark Newman's website on Computational Physics textbook.

  \begin{itemize}
  \tightlist
  \item
    Go to the programs and data page.
  \item
    Open dropped.py, copy and paste the code into your notebook
  \item
    Run the code in your notebook. Try at least 3 value of height and
    time. Copy the codes to 3 boxes at least.

    \begin{itemize}
    \tightlist
    \item
      1: The object is still in flight
    \item
      2: The object passed through the ground
    \item
      3: The object landed on the ground. Output \textasciitilde{} 0
      (doesn't have to be exact)
    \end{itemize}
  \end{itemize}
\end{enumerate}

    \begin{tcolorbox}[breakable, size=fbox, boxrule=1pt, pad at break*=1mm,colback=cellbackground, colframe=cellborder]
\prompt{In}{incolor}{4}{\boxspacing}
\begin{Verbatim}[commandchars=\\\{\}]
\PY{n}{h} \PY{o}{=} \PY{n+nb}{float}\PY{p}{(}\PY{n+nb}{input}\PY{p}{(}\PY{l+s+s2}{\PYZdq{}}\PY{l+s+s2}{Enter the height of the tower: }\PY{l+s+s2}{\PYZdq{}}\PY{p}{)}\PY{p}{)}
\PY{n}{t} \PY{o}{=} \PY{n+nb}{float}\PY{p}{(}\PY{n+nb}{input}\PY{p}{(}\PY{l+s+s2}{\PYZdq{}}\PY{l+s+s2}{Enter the time interval: }\PY{l+s+s2}{\PYZdq{}}\PY{p}{)}\PY{p}{)}

\PY{n}{s} \PY{o}{=} \PY{l+m+mf}{9.81}\PY{o}{*}\PY{n}{t}\PY{o}{*}\PY{o}{*}\PY{l+m+mi}{2}\PY{o}{/}\PY{l+m+mi}{2}

\PY{n+nb}{print}\PY{p}{(}\PY{l+s+s2}{\PYZdq{}}\PY{l+s+s2}{The height of the ball is}\PY{l+s+s2}{\PYZdq{}}\PY{p}{,}\PY{n}{h}\PY{o}{\PYZhy{}}\PY{n}{s}\PY{p}{,}\PY{l+s+s2}{\PYZdq{}}\PY{l+s+s2}{meters}\PY{l+s+s2}{\PYZdq{}}\PY{p}{)}
\end{Verbatim}
\end{tcolorbox}

    \begin{Verbatim}[commandchars=\\\{\}]
Enter the height of the tower:  30
Enter the time interval:  2
    \end{Verbatim}

    \begin{Verbatim}[commandchars=\\\{\}]
The height of the ball is 10.379999999999999 meters
    \end{Verbatim}

    \begin{tcolorbox}[breakable, size=fbox, boxrule=1pt, pad at break*=1mm,colback=cellbackground, colframe=cellborder]
\prompt{In}{incolor}{5}{\boxspacing}
\begin{Verbatim}[commandchars=\\\{\}]
\PY{n}{h} \PY{o}{=} \PY{n+nb}{float}\PY{p}{(}\PY{n+nb}{input}\PY{p}{(}\PY{l+s+s2}{\PYZdq{}}\PY{l+s+s2}{Enter the height of the tower: }\PY{l+s+s2}{\PYZdq{}}\PY{p}{)}\PY{p}{)}
\PY{n}{t} \PY{o}{=} \PY{n+nb}{float}\PY{p}{(}\PY{n+nb}{input}\PY{p}{(}\PY{l+s+s2}{\PYZdq{}}\PY{l+s+s2}{Enter the time interval: }\PY{l+s+s2}{\PYZdq{}}\PY{p}{)}\PY{p}{)}

\PY{n}{s} \PY{o}{=} \PY{l+m+mf}{9.81}\PY{o}{*}\PY{n}{t}\PY{o}{*}\PY{o}{*}\PY{l+m+mi}{2}\PY{o}{/}\PY{l+m+mi}{2}

\PY{n+nb}{print}\PY{p}{(}\PY{l+s+s2}{\PYZdq{}}\PY{l+s+s2}{The height of the ball is}\PY{l+s+s2}{\PYZdq{}}\PY{p}{,}\PY{n}{h}\PY{o}{\PYZhy{}}\PY{n}{s}\PY{p}{,}\PY{l+s+s2}{\PYZdq{}}\PY{l+s+s2}{meters}\PY{l+s+s2}{\PYZdq{}}\PY{p}{)}
\end{Verbatim}
\end{tcolorbox}

    \begin{Verbatim}[commandchars=\\\{\}]
Enter the height of the tower:  40
Enter the time interval:  1
    \end{Verbatim}

    \begin{Verbatim}[commandchars=\\\{\}]
The height of the ball is 35.095 meters
    \end{Verbatim}

    \begin{tcolorbox}[breakable, size=fbox, boxrule=1pt, pad at break*=1mm,colback=cellbackground, colframe=cellborder]
\prompt{In}{incolor}{6}{\boxspacing}
\begin{Verbatim}[commandchars=\\\{\}]
\PY{n}{h} \PY{o}{=} \PY{n+nb}{float}\PY{p}{(}\PY{n+nb}{input}\PY{p}{(}\PY{l+s+s2}{\PYZdq{}}\PY{l+s+s2}{Enter the height of the tower: }\PY{l+s+s2}{\PYZdq{}}\PY{p}{)}\PY{p}{)}
\PY{n}{t} \PY{o}{=} \PY{n+nb}{float}\PY{p}{(}\PY{n+nb}{input}\PY{p}{(}\PY{l+s+s2}{\PYZdq{}}\PY{l+s+s2}{Enter the time interval: }\PY{l+s+s2}{\PYZdq{}}\PY{p}{)}\PY{p}{)}

\PY{n}{s} \PY{o}{=} \PY{l+m+mf}{9.81}\PY{o}{*}\PY{n}{t}\PY{o}{*}\PY{o}{*}\PY{l+m+mi}{2}\PY{o}{/}\PY{l+m+mi}{2}

\PY{c+c1}{\PYZsh{} h = 50}
\PY{c+c1}{\PYZsh{} t = 3.1919}
\PY{n+nb}{print}\PY{p}{(}\PY{l+s+s2}{\PYZdq{}}\PY{l+s+s2}{The height of the ball is}\PY{l+s+s2}{\PYZdq{}}\PY{p}{,}\PY{n}{h}\PY{o}{\PYZhy{}}\PY{n}{s}\PY{p}{,}\PY{l+s+s2}{\PYZdq{}}\PY{l+s+s2}{meters}\PY{l+s+s2}{\PYZdq{}}\PY{p}{)}
\end{Verbatim}
\end{tcolorbox}

    \begin{Verbatim}[commandchars=\\\{\}]
Enter the height of the tower:  50
Enter the time interval:  3.1919
    \end{Verbatim}

    \begin{Verbatim}[commandchars=\\\{\}]
The height of the ball is 0.026753382949998183 meters
    \end{Verbatim}

    \begin{enumerate}
\def\labelenumi{\arabic{enumi}.}
\setcounter{enumi}{2}
\tightlist
\item
  Exercise 2.1
\end{enumerate}

    \begin{tcolorbox}[breakable, size=fbox, boxrule=1pt, pad at break*=1mm,colback=cellbackground, colframe=cellborder]
\prompt{In}{incolor}{8}{\boxspacing}
\begin{Verbatim}[commandchars=\\\{\}]
\PY{k+kn}{import} \PY{n+nn}{numpy} \PY{k}{as} \PY{n+nn}{np}

\PY{n}{height} \PY{o}{=} \PY{n+nb}{float}\PY{p}{(}\PY{n+nb}{input}\PY{p}{(}\PY{l+s+s2}{\PYZdq{}}\PY{l+s+s2}{Enter the height of the tower in meters: }\PY{l+s+s2}{\PYZdq{}}\PY{p}{)}\PY{p}{)}

\PY{c+c1}{\PYZsh{} We can use the following formula to calculate the}
\PY{c+c1}{\PYZsh{} height of the ball with variables gravity and time:}
\PY{c+c1}{\PYZsh{} h = 1/2gt\PYZca{}2}
\PY{c+c1}{\PYZsh{} Solving for time, we get:}
\PY{c+c1}{\PYZsh{} t = sqrt(2h/g)}
\PY{n}{gravity} \PY{o}{=} \PY{l+m+mf}{9.8}
\PY{n}{time} \PY{o}{=} \PY{n}{np}\PY{o}{.}\PY{n}{sqrt}\PY{p}{(}\PY{p}{(}\PY{l+m+mi}{2}\PY{o}{*} \PY{n}{height}\PY{p}{)} \PY{o}{/} \PY{n}{gravity}\PY{p}{)}

\PY{n+nb}{print}\PY{p}{(}\PY{l+s+sa}{f}\PY{l+s+s2}{\PYZdq{}}\PY{l+s+s2}{The ball takes approximately }\PY{l+s+si}{\PYZob{}}\PY{n}{time}\PY{l+s+si}{\PYZcb{}}\PY{l+s+s2}{ seconds to hit the ground.}\PY{l+s+s2}{\PYZdq{}}\PY{p}{)}
\end{Verbatim}
\end{tcolorbox}

    \begin{Verbatim}[commandchars=\\\{\}]
Enter the height of the tower in meters:  100
    \end{Verbatim}

    \begin{Verbatim}[commandchars=\\\{\}]
The ball takes approximately 4.5175395145262565 seconds to hit the ground.
    \end{Verbatim}

    \begin{enumerate}
\def\labelenumi{\arabic{enumi}.}
\setcounter{enumi}{3}
\tightlist
\item
  Exercise 2.2
\end{enumerate}

    \begin{tcolorbox}[breakable, size=fbox, boxrule=1pt, pad at break*=1mm,colback=cellbackground, colframe=cellborder]
\prompt{In}{incolor}{10}{\boxspacing}
\begin{Verbatim}[commandchars=\\\{\}]
\PY{c+c1}{\PYZsh{} PART B}
\PY{c+c1}{\PYZsh{} The formula they gave us is:}
\PY{c+c1}{\PYZsh{} h = ((G * M * T**2) / (4 * pi**2))**(1/3) \PYZhy{} R}
\PY{n}{G} \PY{o}{=} \PY{l+m+mf}{6.67} \PY{o}{*} \PY{l+m+mi}{10}\PY{o}{*}\PY{o}{*}\PY{o}{\PYZhy{}}\PY{l+m+mi}{11}
\PY{n}{M} \PY{o}{=} \PY{l+m+mf}{5.97} \PY{o}{*} \PY{l+m+mi}{10}\PY{o}{*}\PY{o}{*}\PY{l+m+mi}{27} \PY{c+c1}{\PYZsh{} Convert kg to g by added 3 to the exponent}
\PY{n}{R} \PY{o}{=} \PY{l+m+mi}{6371} \PY{o}{*} \PY{l+m+mi}{1000} \PY{c+c1}{\PYZsh{} Convert km to m by multiplying by 1000}

\PY{n}{T} \PY{o}{=} \PY{n+nb}{float}\PY{p}{(}\PY{n+nb}{input}\PY{p}{(}\PY{l+s+s2}{\PYZdq{}}\PY{l+s+s2}{Enter your desired value of T: }\PY{l+s+s2}{\PYZdq{}}\PY{p}{)}\PY{p}{)}
\PY{n}{height} \PY{o}{=} \PY{p}{(}\PY{p}{(}\PY{n}{G} \PY{o}{*} \PY{n}{M} \PY{o}{*} \PY{n}{T}\PY{o}{*}\PY{o}{*}\PY{l+m+mi}{2}\PY{p}{)} \PY{o}{/} \PY{p}{(}\PY{l+m+mi}{4} \PY{o}{*} \PY{n}{np}\PY{o}{.}\PY{n}{pi}\PY{o}{*}\PY{o}{*}\PY{l+m+mi}{2}\PY{p}{)}\PY{p}{)}\PY{o}{*}\PY{o}{*}\PY{p}{(}\PY{l+m+mi}{1}\PY{o}{/}\PY{l+m+mi}{3}\PY{p}{)} \PY{o}{\PYZhy{}} \PY{n}{R}
\PY{n+nb}{print}\PY{p}{(}\PY{l+s+sa}{f}\PY{l+s+s2}{\PYZdq{}}\PY{l+s+s2}{For a satellite to orbit the Earth once every }\PY{l+s+si}{\PYZob{}}\PY{n}{T}\PY{l+s+si}{\PYZcb{}}\PY{l+s+s2}{ }\PY{l+s+se}{\PYZbs{}}
\PY{l+s+s2}{seconds it needs to be launched to a height of }\PY{l+s+se}{\PYZbs{}}
\PY{l+s+s2}{approximately }\PY{l+s+si}{\PYZob{}}\PY{n}{height}\PY{l+s+si}{\PYZcb{}}\PY{l+s+s2}{ meters above the Earth}\PY{l+s+s2}{\PYZsq{}}\PY{l+s+s2}{s surface.}\PY{l+s+s2}{\PYZdq{}}\PY{p}{)}
\end{Verbatim}
\end{tcolorbox}

    \begin{Verbatim}[commandchars=\\\{\}]
Enter your desired value of T:  555555555
    \end{Verbatim}

    \begin{Verbatim}[commandchars=\\\{\}]
For a satellite to orbit the Earth once every 555555555.0 seconds it needs to be
launched to a height of approximately 146008965087.80573 meters above the
Earth's surface.
    \end{Verbatim}

    \begin{tcolorbox}[breakable, size=fbox, boxrule=1pt, pad at break*=1mm,colback=cellbackground, colframe=cellborder]
\prompt{In}{incolor}{11}{\boxspacing}
\begin{Verbatim}[commandchars=\\\{\}]
\PY{c+c1}{\PYZsh{} PART C}
\PY{n}{T} \PY{o}{=} \PY{l+m+mi}{86400} \PY{c+c1}{\PYZsh{} Seconds per day}
\PY{n}{height\PYZus{}day} \PY{o}{=} \PY{p}{(}\PY{p}{(}\PY{n}{G} \PY{o}{*} \PY{n}{M} \PY{o}{*} \PY{n}{T}\PY{o}{*}\PY{o}{*}\PY{l+m+mi}{2}\PY{p}{)} \PY{o}{/} \PY{p}{(}\PY{l+m+mi}{4} \PY{o}{*} \PY{n}{np}\PY{o}{.}\PY{n}{pi}\PY{o}{*}\PY{o}{*}\PY{l+m+mi}{2}\PY{p}{)}\PY{p}{)}\PY{o}{*}\PY{o}{*}\PY{p}{(}\PY{l+m+mi}{1}\PY{o}{/}\PY{l+m+mi}{3}\PY{p}{)} \PY{o}{\PYZhy{}} \PY{n}{R}
\PY{n+nb}{print}\PY{p}{(}\PY{l+s+sa}{f}\PY{l+s+s2}{\PYZdq{}}\PY{l+s+s2}{For a satellite to orbit the Earth once per day it }\PY{l+s+se}{\PYZbs{}}
\PY{l+s+s2}{needs to be launched to a height of }\PY{l+s+se}{\PYZbs{}}
\PY{l+s+s2}{approximately }\PY{l+s+si}{\PYZob{}}\PY{n}{height\PYZus{}day}\PY{l+s+si}{\PYZcb{}}\PY{l+s+s2}{ meters above the Earth}\PY{l+s+s2}{\PYZsq{}}\PY{l+s+s2}{s surface.}\PY{l+s+s2}{\PYZdq{}}\PY{p}{)}

\PY{n}{T} \PY{o}{=} \PY{l+m+mi}{5400} \PY{c+c1}{\PYZsh{} Seconds per 90 minutes}
\PY{n}{height} \PY{o}{=} \PY{p}{(}\PY{p}{(}\PY{n}{G} \PY{o}{*} \PY{n}{M} \PY{o}{*} \PY{n}{T}\PY{o}{*}\PY{o}{*}\PY{l+m+mi}{2}\PY{p}{)} \PY{o}{/} \PY{p}{(}\PY{l+m+mi}{4} \PY{o}{*} \PY{n}{np}\PY{o}{.}\PY{n}{pi}\PY{o}{*}\PY{o}{*}\PY{l+m+mi}{2}\PY{p}{)}\PY{p}{)}\PY{o}{*}\PY{o}{*}\PY{p}{(}\PY{l+m+mi}{1}\PY{o}{/}\PY{l+m+mi}{3}\PY{p}{)} \PY{o}{\PYZhy{}} \PY{n}{R}
\PY{n+nb}{print}\PY{p}{(}\PY{l+s+sa}{f}\PY{l+s+s2}{\PYZdq{}}\PY{l+s+s2}{For a satellite to orbit the Earth once per 90 minutes }\PY{l+s+se}{\PYZbs{}}
\PY{l+s+s2}{needs to be launched to a height of }\PY{l+s+se}{\PYZbs{}}
\PY{l+s+s2}{approximately }\PY{l+s+si}{\PYZob{}}\PY{n}{height}\PY{l+s+si}{\PYZcb{}}\PY{l+s+s2}{ meters above the Earth}\PY{l+s+s2}{\PYZsq{}}\PY{l+s+s2}{s surface.}\PY{l+s+s2}{\PYZdq{}}\PY{p}{)}

\PY{n}{T} \PY{o}{=} \PY{l+m+mi}{2700} \PY{c+c1}{\PYZsh{} Seconds per 45 minutes}
\PY{n}{height} \PY{o}{=} \PY{p}{(}\PY{p}{(}\PY{n}{G} \PY{o}{*} \PY{n}{M} \PY{o}{*} \PY{n}{T}\PY{o}{*}\PY{o}{*}\PY{l+m+mi}{2}\PY{p}{)} \PY{o}{/} \PY{p}{(}\PY{l+m+mi}{4} \PY{o}{*} \PY{n}{np}\PY{o}{.}\PY{n}{pi}\PY{o}{*}\PY{o}{*}\PY{l+m+mi}{2}\PY{p}{)}\PY{p}{)}\PY{o}{*}\PY{o}{*}\PY{p}{(}\PY{l+m+mi}{1}\PY{o}{/}\PY{l+m+mi}{3}\PY{p}{)} \PY{o}{\PYZhy{}} \PY{n}{R}
\PY{n+nb}{print}\PY{p}{(}\PY{l+s+sa}{f}\PY{l+s+s2}{\PYZdq{}}\PY{l+s+s2}{For a satellite to orbit the Earth once per 45 minutes }\PY{l+s+se}{\PYZbs{}}
\PY{l+s+s2}{needs to be launched to a height of }\PY{l+s+se}{\PYZbs{}}
\PY{l+s+s2}{approximately }\PY{l+s+si}{\PYZob{}}\PY{n}{height}\PY{l+s+si}{\PYZcb{}}\PY{l+s+s2}{ meters above the Earth}\PY{l+s+s2}{\PYZsq{}}\PY{l+s+s2}{s surface.}\PY{l+s+s2}{\PYZdq{}}\PY{p}{)}
\end{Verbatim}
\end{tcolorbox}

    \begin{Verbatim}[commandchars=\\\{\}]
For a satellite to orbit the Earth once per day it needs to be launched to a
height of approximately 415898101.7617496 meters above the Earth's surface.
For a satellite to orbit the Earth once per 90 minutes needs to be launched to a
height of approximately 60132216.25372859 meters above the Earth's surface.
For a satellite to orbit the Earth once per 45 minutes needs to be launched to a
height of approximately 35523401.02189176 meters above the Earth's surface.
    \end{Verbatim}

    The conclusion that I can draw from the last calculation is that as the
orbital period decreases so does the altitude at which the satellite
must orbit. In other words, in order for a satellite to orbit the Earth
faster it must be closer to Earth's surface.

    \begin{tcolorbox}[breakable, size=fbox, boxrule=1pt, pad at break*=1mm,colback=cellbackground, colframe=cellborder]
\prompt{In}{incolor}{13}{\boxspacing}
\begin{Verbatim}[commandchars=\\\{\}]
\PY{c+c1}{\PYZsh{} PART D}
\PY{n}{T} \PY{o}{=} \PY{l+m+mi}{86148} \PY{c+c1}{\PYZsh{} Seconds per sidereal day}
\PY{n}{height\PYZus{}sidereal\PYZus{}day} \PY{o}{=} \PY{p}{(}\PY{p}{(}\PY{n}{G} \PY{o}{*} \PY{n}{M} \PY{o}{*} \PY{n}{T}\PY{o}{*}\PY{o}{*}\PY{l+m+mi}{2}\PY{p}{)} \PY{o}{/} \PY{p}{(}\PY{l+m+mi}{4} \PY{o}{*} \PY{n}{np}\PY{o}{.}\PY{n}{pi}\PY{o}{*}\PY{o}{*}\PY{l+m+mi}{2}\PY{p}{)}\PY{p}{)}\PY{o}{*}\PY{o}{*}\PY{p}{(}\PY{l+m+mi}{1}\PY{o}{/}\PY{l+m+mi}{3}\PY{p}{)} \PY{o}{\PYZhy{}} \PY{n}{R}
\PY{n}{difference} \PY{o}{=} \PY{n}{height\PYZus{}day} \PY{o}{\PYZhy{}} \PY{n}{height\PYZus{}sidereal\PYZus{}day}
\PY{n+nb}{print}\PY{p}{(}\PY{l+s+sa}{f}\PY{l+s+s2}{\PYZdq{}}\PY{l+s+s2}{The difference in altitude between a day orbit vs a sideral day orbit is }\PY{l+s+si}{\PYZob{}}\PY{n}{difference}\PY{l+s+si}{\PYZcb{}}\PY{l+s+s2}{ meters.}\PY{l+s+s2}{\PYZdq{}}\PY{p}{)}
\end{Verbatim}
\end{tcolorbox}

    \begin{Verbatim}[commandchars=\\\{\}]
The difference in altitude between a day orbit vs a sideral day orbit is
821478.4627931714 meters.
    \end{Verbatim}

    \begin{enumerate}
\def\labelenumi{\arabic{enumi}.}
\setcounter{enumi}{4}
\tightlist
\item
  Exercise 2.3
\end{enumerate}

    \begin{tcolorbox}[breakable, size=fbox, boxrule=1pt, pad at break*=1mm,colback=cellbackground, colframe=cellborder]
\prompt{In}{incolor}{15}{\boxspacing}
\begin{Verbatim}[commandchars=\\\{\}]
\PY{n}{x} \PY{o}{=} \PY{n+nb}{float}\PY{p}{(}\PY{n+nb}{input}\PY{p}{(}\PY{l+s+s2}{\PYZdq{}}\PY{l+s+s2}{Enter a value for the x\PYZhy{}coordinate: }\PY{l+s+s2}{\PYZdq{}}\PY{p}{)}\PY{p}{)}
\PY{n}{y} \PY{o}{=} \PY{n+nb}{float}\PY{p}{(}\PY{n+nb}{input}\PY{p}{(}\PY{l+s+s2}{\PYZdq{}}\PY{l+s+s2}{Enter a value for the y\PYZhy{}coordinate: }\PY{l+s+s2}{\PYZdq{}}\PY{p}{)}\PY{p}{)}

\PY{c+c1}{\PYZsh{} To calculate r given x and y we can use the following formula:}
\PY{n}{r} \PY{o}{=} \PY{n}{np}\PY{o}{.}\PY{n}{sqrt}\PY{p}{(}\PY{n}{x}\PY{o}{*}\PY{o}{*}\PY{l+m+mi}{2} \PY{o}{+} \PY{n}{y}\PY{o}{*}\PY{o}{*}\PY{l+m+mi}{2}\PY{p}{)}

\PY{c+c1}{\PYZsh{} To calculate theta given x and y we can use the following formula:}
\PY{n}{theta} \PY{o}{=} \PY{n}{np}\PY{o}{.}\PY{n}{degrees}\PY{p}{(}\PY{n}{np}\PY{o}{.}\PY{n}{arctan2}\PY{p}{(}\PY{n}{y}\PY{p}{,} \PY{n}{x}\PY{p}{)}\PY{p}{)}

\PY{n+nb}{print}\PY{p}{(}\PY{l+s+sa}{f}\PY{l+s+s2}{\PYZdq{}}\PY{l+s+s2}{The corresponding polar coordinates are:}\PY{l+s+se}{\PYZbs{}n}\PY{l+s+s2}{r = }\PY{l+s+si}{\PYZob{}}\PY{n}{r}\PY{l+s+si}{\PYZcb{}}\PY{l+s+se}{\PYZbs{}n}\PY{l+s+s2}{theta = }\PY{l+s+si}{\PYZob{}}\PY{n}{theta}\PY{l+s+si}{\PYZcb{}}\PY{l+s+s2}{\PYZdq{}}\PY{p}{)}
\end{Verbatim}
\end{tcolorbox}

    \begin{Verbatim}[commandchars=\\\{\}]
Enter a value for the x-coordinate:  5
Enter a value for the y-coordinate:  4
    \end{Verbatim}

    \begin{Verbatim}[commandchars=\\\{\}]
The corresponding polar coordinates are:
r = 6.4031242374328485
theta = 38.65980825409009
    \end{Verbatim}

    \begin{enumerate}
\def\labelenumi{\arabic{enumi}.}
\setcounter{enumi}{5}
\tightlist
\item
  Exercise 2.6
\end{enumerate}

    \begin{tcolorbox}[breakable, size=fbox, boxrule=1pt, pad at break*=1mm,colback=cellbackground, colframe=cellborder]
\prompt{In}{incolor}{17}{\boxspacing}
\begin{Verbatim}[commandchars=\\\{\}]
\PY{n}{l1} \PY{o}{=} \PY{n+nb}{float}\PY{p}{(}\PY{n+nb}{input}\PY{p}{(}\PY{l+s+s2}{\PYZdq{}}\PY{l+s+s2}{Enter the distance to the Sun at perihelion (l1): }\PY{l+s+s2}{\PYZdq{}}\PY{p}{)}\PY{p}{)}
\PY{n}{v1} \PY{o}{=} \PY{n+nb}{float}\PY{p}{(}\PY{n+nb}{input}\PY{p}{(}\PY{l+s+s2}{\PYZdq{}}\PY{l+s+s2}{Enter the velocity at perhelion (v1): }\PY{l+s+s2}{\PYZdq{}}\PY{p}{)}\PY{p}{)}
\PY{n+nb}{print}\PY{p}{(}\PY{l+s+sa}{f}\PY{l+s+s2}{\PYZdq{}}\PY{l+s+s2}{l1 = }\PY{l+s+si}{\PYZob{}}\PY{n}{l1}\PY{l+s+si}{\PYZcb{}}\PY{l+s+se}{\PYZbs{}n}\PY{l+s+s2}{v1 = }\PY{l+s+si}{\PYZob{}}\PY{n}{v1}\PY{l+s+si}{\PYZcb{}}\PY{l+s+s2}{\PYZdq{}}\PY{p}{)}

\PY{c+c1}{\PYZsh{} Given l1 and v1 we can solve for v2 by using the following formula:}
\PY{c+c1}{\PYZsh{} v2\PYZca{}(2)\PYZhy{}((2GM))/((v1)l1)v2\PYZhy{}[v1\PYZca{}(2)\PYZhy{}((2GM))/(l1)]=0}
\PY{c+c1}{\PYZsh{} Simplifying we get:}
\PY{c+c1}{\PYZsh{} v2 = v1}
\PY{n}{v2} \PY{o}{=} \PY{n}{v1}
\PY{n+nb}{print}\PY{p}{(}\PY{l+s+sa}{f}\PY{l+s+s2}{\PYZdq{}}\PY{l+s+s2}{v2 = }\PY{l+s+si}{\PYZob{}}\PY{n}{v2}\PY{l+s+si}{\PYZcb{}}\PY{l+s+s2}{\PYZdq{}}\PY{p}{)}

\PY{c+c1}{\PYZsh{} Now that we have v2 we can calculate l2 by the following formula:}
\PY{c+c1}{\PYZsh{} l2 = l1(v1)/v2}
\PY{n}{l2} \PY{o}{=} \PY{n}{l1}\PY{o}{*}\PY{n}{v1}\PY{o}{/}\PY{n}{v2}
\PY{n+nb}{print}\PY{p}{(}\PY{l+s+sa}{f}\PY{l+s+s2}{\PYZdq{}}\PY{l+s+s2}{l2 = }\PY{l+s+si}{\PYZob{}}\PY{n}{l2}\PY{l+s+si}{\PYZcb{}}\PY{l+s+s2}{\PYZdq{}}\PY{p}{)}

\PY{c+c1}{\PYZsh{} Now that we have l1 and l2 we can calculate the Semi\PYZhy{}major axis:}
\PY{n}{semi\PYZus{}major\PYZus{}axis} \PY{o}{=} \PY{p}{(}\PY{l+m+mi}{1}\PY{o}{/}\PY{l+m+mi}{2}\PY{p}{)}\PY{o}{*}\PY{p}{(}\PY{n}{l1}\PY{o}{+}\PY{n}{l2}\PY{p}{)}
\PY{n+nb}{print}\PY{p}{(}\PY{l+s+sa}{f}\PY{l+s+s2}{\PYZdq{}}\PY{l+s+s2}{semi major axis = }\PY{l+s+si}{\PYZob{}}\PY{n}{semi\PYZus{}major\PYZus{}axis}\PY{l+s+si}{\PYZcb{}}\PY{l+s+s2}{\PYZdq{}}\PY{p}{)}

\PY{n}{semi\PYZus{}minor\PYZus{}axis} \PY{o}{=} \PY{n}{np}\PY{o}{.}\PY{n}{sqrt}\PY{p}{(}\PY{p}{(}\PY{n}{l1}\PY{o}{*}\PY{n}{l2}\PY{p}{)}\PY{p}{)}
\PY{n+nb}{print}\PY{p}{(}\PY{l+s+sa}{f}\PY{l+s+s2}{\PYZdq{}}\PY{l+s+s2}{semi minor axis = }\PY{l+s+si}{\PYZob{}}\PY{n}{semi\PYZus{}minor\PYZus{}axis}\PY{l+s+si}{\PYZcb{}}\PY{l+s+s2}{\PYZdq{}}\PY{p}{)}
\PY{n+nb}{print}\PY{p}{(}\PY{l+s+sa}{f}\PY{l+s+s2}{\PYZdq{}}\PY{l+s+s2}{pi = }\PY{l+s+si}{\PYZob{}}\PY{n}{np}\PY{o}{.}\PY{n}{pi}\PY{l+s+si}{\PYZcb{}}\PY{l+s+s2}{\PYZdq{}}\PY{p}{)}

\PY{n}{orbital\PYZus{}period} \PY{o}{=} \PY{p}{(}\PY{l+m+mi}{2} \PY{o}{*} \PY{n}{np}\PY{o}{.}\PY{n}{pi} \PY{o}{*} \PY{n}{semi\PYZus{}major\PYZus{}axis} \PY{o}{*} \PY{n}{semi\PYZus{}minor\PYZus{}axis}\PY{p}{)}\PY{o}{/}\PY{p}{(}\PY{n}{l1}\PY{o}{*}\PY{n}{v1}\PY{p}{)}
\PY{n+nb}{print}\PY{p}{(}\PY{l+s+sa}{f}\PY{l+s+s2}{\PYZdq{}}\PY{l+s+s2}{orbital period = }\PY{l+s+si}{\PYZob{}}\PY{n}{orbital\PYZus{}period}\PY{l+s+si}{\PYZcb{}}\PY{l+s+s2}{ seconds, or }\PY{l+s+si}{\PYZob{}}\PY{n}{orbital\PYZus{}period}\PY{o}{/}\PY{p}{(}\PY{l+m+mf}{3.154}\PY{o}{*}\PY{l+m+mi}{10}\PY{o}{*}\PY{o}{*}\PY{l+m+mi}{7}\PY{p}{)}\PY{l+s+si}{\PYZcb{}}\PY{l+s+s2}{ years.}\PY{l+s+s2}{\PYZdq{}}\PY{p}{)}

\PY{n}{orbital\PYZus{}eccentricity} \PY{o}{=} \PY{p}{(}\PY{n}{l2} \PY{o}{\PYZhy{}} \PY{n}{l1}\PY{p}{)}\PY{o}{/}\PY{p}{(}\PY{n}{l2} \PY{o}{+} \PY{n}{l1}\PY{p}{)}
\PY{n+nb}{print}\PY{p}{(}\PY{l+s+sa}{f}\PY{l+s+s2}{\PYZdq{}}\PY{l+s+s2}{orbital eccentricity = }\PY{l+s+si}{\PYZob{}}\PY{n}{orbital\PYZus{}eccentricity}\PY{l+s+si}{\PYZcb{}}\PY{l+s+s2}{\PYZdq{}}\PY{p}{)}
\end{Verbatim}
\end{tcolorbox}

    \begin{Verbatim}[commandchars=\\\{\}]
Enter the distance to the Sun at perihelion (l1):  147100000000
Enter the velocity at perhelion (v1):  30287
    \end{Verbatim}

    \begin{Verbatim}[commandchars=\\\{\}]
l1 = 147100000000.0
v1 = 30287.0
v2 = 30287.0
l2 = 147100000000.0
semi major axis = 147100000000.0
semi minor axis = 147100000000.0
pi = 3.141592653589793
orbital period = 30516609.72318543 seconds, or 0.9675526228023282 years.
orbital eccentricity = 0.0
    \end{Verbatim}

    \begin{enumerate}
\def\labelenumi{\arabic{enumi}.}
\setcounter{enumi}{6}
\tightlist
\item
  Exercise 2.7
\end{enumerate}

    \begin{tcolorbox}[breakable, size=fbox, boxrule=1pt, pad at break*=1mm,colback=cellbackground, colframe=cellborder]
\prompt{In}{incolor}{19}{\boxspacing}
\begin{Verbatim}[commandchars=\\\{\}]
\PY{n}{cat\PYZus{}nums} \PY{o}{=} \PY{p}{[}\PY{l+m+mi}{1}\PY{p}{]}
\PY{n}{cat\PYZus{}num} \PY{o}{=} \PY{l+m+mi}{1}
\PY{n}{limit} \PY{o}{=} \PY{l+m+mi}{1000000000}

\PY{k}{while} \PY{p}{(}\PY{n}{cat\PYZus{}num} \PY{o}{\PYZlt{}}\PY{o}{=} \PY{n}{limit}\PY{p}{)}\PY{p}{:}
    \PY{n}{n} \PY{o}{=} \PY{n+nb}{len}\PY{p}{(}\PY{n}{cat\PYZus{}nums}\PY{p}{)} \PY{o}{\PYZhy{}} \PY{l+m+mi}{1}
    \PY{n}{cat\PYZus{}num} \PY{o}{=} \PY{p}{(}\PY{p}{(}\PY{l+m+mi}{4} \PY{o}{*} \PY{n}{n} \PY{o}{+} \PY{l+m+mi}{2}\PY{p}{)}\PY{o}{/}\PY{p}{(}\PY{n}{n} \PY{o}{+} \PY{l+m+mi}{2}\PY{p}{)}\PY{p}{)}\PY{o}{*}\PY{n}{cat\PYZus{}nums}\PY{p}{[}\PY{n}{n}\PY{p}{]}
    \PY{k}{if} \PY{p}{(}\PY{n}{cat\PYZus{}num} \PY{o}{\PYZgt{}} \PY{n}{limit}\PY{p}{)}\PY{p}{:}
        \PY{k}{break}
    \PY{n}{cat\PYZus{}nums}\PY{o}{.}\PY{n}{append}\PY{p}{(}\PY{n}{cat\PYZus{}num}\PY{p}{)}
    \PY{n+nb}{print}\PY{p}{(}\PY{n}{cat\PYZus{}nums}\PY{p}{)}
\end{Verbatim}
\end{tcolorbox}

    \begin{Verbatim}[commandchars=\\\{\}]
[1, 1.0]
[1, 1.0, 2.0]
[1, 1.0, 2.0, 5.0]
[1, 1.0, 2.0, 5.0, 14.0]
[1, 1.0, 2.0, 5.0, 14.0, 42.0]
[1, 1.0, 2.0, 5.0, 14.0, 42.0, 132.0]
[1, 1.0, 2.0, 5.0, 14.0, 42.0, 132.0, 429.0]
[1, 1.0, 2.0, 5.0, 14.0, 42.0, 132.0, 429.0, 1430.0]
[1, 1.0, 2.0, 5.0, 14.0, 42.0, 132.0, 429.0, 1430.0, 4862.0]
[1, 1.0, 2.0, 5.0, 14.0, 42.0, 132.0, 429.0, 1430.0, 4862.0, 16796.0]
[1, 1.0, 2.0, 5.0, 14.0, 42.0, 132.0, 429.0, 1430.0, 4862.0, 16796.0, 58786.0]
[1, 1.0, 2.0, 5.0, 14.0, 42.0, 132.0, 429.0, 1430.0, 4862.0, 16796.0, 58786.0,
208012.0]
[1, 1.0, 2.0, 5.0, 14.0, 42.0, 132.0, 429.0, 1430.0, 4862.0, 16796.0, 58786.0,
208012.0, 742900.0]
[1, 1.0, 2.0, 5.0, 14.0, 42.0, 132.0, 429.0, 1430.0, 4862.0, 16796.0, 58786.0,
208012.0, 742900.0, 2674440.0]
[1, 1.0, 2.0, 5.0, 14.0, 42.0, 132.0, 429.0, 1430.0, 4862.0, 16796.0, 58786.0,
208012.0, 742900.0, 2674440.0, 9694845.0]
[1, 1.0, 2.0, 5.0, 14.0, 42.0, 132.0, 429.0, 1430.0, 4862.0, 16796.0, 58786.0,
208012.0, 742900.0, 2674440.0, 9694845.0, 35357670.0]
[1, 1.0, 2.0, 5.0, 14.0, 42.0, 132.0, 429.0, 1430.0, 4862.0, 16796.0, 58786.0,
208012.0, 742900.0, 2674440.0, 9694845.0, 35357670.0, 129644790.0]
[1, 1.0, 2.0, 5.0, 14.0, 42.0, 132.0, 429.0, 1430.0, 4862.0, 16796.0, 58786.0,
208012.0, 742900.0, 2674440.0, 9694845.0, 35357670.0, 129644790.0, 477638700.0]
    \end{Verbatim}

    \begin{tcolorbox}[breakable, size=fbox, boxrule=1pt, pad at break*=1mm,colback=cellbackground, colframe=cellborder]
\prompt{In}{incolor}{ }{\boxspacing}
\begin{Verbatim}[commandchars=\\\{\}]

\end{Verbatim}
\end{tcolorbox}


    % Add a bibliography block to the postdoc
    
    
    
\end{document}
